\begin{itemize}
\item[\textbf{4.1}]\emph{ Provea dos ejemplos de programaci\'on en donde multi-hebras provee un mejor rendimiento que una solucion de una hebra}\\
R:
	\begin{enumerate}	
		\item Un servidor Web que procese cada petici\'on en threads separados.
		\item Un programa GUI interactivo, por ejemplo, un debugger, donde un thread es usado para monitorear inputs, otro thread representa las aplicaciones corriendo y un tercer thread monitorea el rendimiento.
	\end{enumerate}
\item[\textbf{4.3}]\emph{ Describa las acciones realizadas por el kernel en un context-switch entre hebras a nivel del kernel }\\
R:\\
El context-switching entre threads del kernel requiere generalmente guardar los valores de los registros de la CPU desde el thread que esta siendo 'switcheado' y recuperando los registros del CPU del nuevo thread que viene en el schedule.
\item[\textbf{4.7}]\emph{ Provea dos ejemplos de programaci\'on en el cual multi-hebras no proveen un mejor rendimiento que una solucion de una hebra}\\
R:
	\begin{enumerate}
		\item Ejemplo 1:\\\\
string filenames[100];\\
/* Se crean threads del 1 al 100 */\\
create\_threads();\\
/* Cada thread crea un archivo en particular, cuyo nombre se pasa al programa ejecutado por el thread */\\
\\while (1)\{ \\
not\_created = FALSE;\\
for i = 0 .. 99 \{ \\
if (not(exists(filename[i])) \{ not\_created = TRUE; \} \\
\} \\
if (not\_created == FALSE)\{ \\
printf("Done!");\\
return;\\
\} \\
\} \\ \\

Este es el seudoc\'odigo de un programa multihebra que no desempe~na mejor rendimiento que un programa de una hebra para solucionar el mismo problema.\\
		\item Ejemplo 2:\\ \\
			Si el tiempo de computaci\'on de los procesos no difiere significativamente, un programa multihebra no es m\'as \'util en comparaci\'on a una soluci\'on de una hebra.\\
	\end{enumerate}
\item[\textbf{4.8}]\emph{ Describa las acciones realizadas por una biblioteca de hebras en un context-switch entre hebras a nivel de usuario }\\
R:\\
Para realizar el context-switch entre hebras a nivel de usuario la biblioteca de threads guarda el contexto del anterior thread en su TCB (Thread Control Block) y carga el contexto del nuevo thread. El TCB contiene un stack de punteros, un Program Counter, valores de los registros y el estado actual del thread.\\
\item[\textbf{4.10}]\emph{ Cuales de los siguientes componenes del estado del programa son compartidos por las hebras en un procesamiento multi-hebra}
	\begin{enumerate}
		\item Valores de Registro
		\item Memoria Heap
		\item Variables Globales
		\item Memoria de Stack
	\end{enumerate}
	R:
	 \begin{itemize}
		\item Heap memory
		\item Global variables
	 \end{itemize}
\item[\textbf{4.11}]\emph{ ?`Puede una soluci\'on multi-hebra usando multiples hebras de niveles de usuario obtener un mejor rendimiento en un sistema con multiprocesador que dentro de uno de 1 solo procesador? Explique.}\\
R:\\
No hay mejor rendimiento, pues el sistema operativo v\'e solo un proceso individual y no va a gestionar los diferentes threads del proceso en procesadores diferentes.\\
\item[\textbf{4.13}]\emph{ El programa muestro a continuaci\'on usa la API Pthreads. ?`Que es lo seria el output del programa en la linea C y linea P?}
	\begin{itemize}
	\item Figura 4.14: Programa en C para el Ejercicio 4.13\\
\#include <stdio.h>\\
\#include <pthread.h>\\
\\
int value = 0;\\
void *runner(void *param); /* the thread */\\
\\
int main(int argc, char *argv[])\\
\{\\
	int pid;\\
	pthread\_t tid;\\
	pthread\_attr\_t attr;\\
	\\
	pid = fork();\\
	\\
	if (pid == 0) \{ /* child process */\\
		pthread\_ttr\_init(\&attr);\\
		pthread\_create(\&tid,\&attr,runner,NULL);\\
		pthread\_join(tid,NULL);\\
		fprintf(``CHILD: valuea = \%d'',value); /* LINE C */\\
	\}\\
	else if (pid > 0) \{ /* parent process */\\
		wait(NULL);\\
		fprintf(``PARENT: valuea = \%d'',value); /* LINE P */\\
	\}\\
	\\
	void *runner(void *param) \{\\
		value = 5;\\
		pthread\_exit(0);\\
	\}\\
\}
	\end{itemize}
R:\\
Linea C, 5.\\
Linea P, 0.\\

\end{itemize}
