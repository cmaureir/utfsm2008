\begin{itemize}
\item[\textbf{3.6}]\emph{ Describa las diferencias entre short-term, medium-term y long-term scheduling}\\
R:
	\begin{itemize}
		\item \textbf{Long Term}: Determina que programas son admitidos para ejecucion y cuales deberan ser salidos. Es invocado pocas veces, segundos o minutos.
		\item \textbf{Medium Term}: Determina que procesos deben ser suspendidos o resumidos, es el que activa el context switch.
		\item \textbf{Short Term}: Determina que procesos estan listos para ocupar CPU y por cuanto tiempo. Invocado frecuentemente, milisegundos.
	\end{itemize}

\item[\textbf{3.7}]\emph{ Describa las acciones realizadas por el kernel para realizar el context-switch entre procesos }\\
R:
	\begin{enumerate}
		\item Proceso $P_{0}$ es interrumpido por llamada de sistema o interrupcion
		\item Salva el estado de $P_{0}$ en $PCB_{0}$
		\item Recupera informacion de proceso $P_{1}$ del $PCB_{1}$
		\item Ejecuta $P_{1}$ desde el estado $PCB_{1}$
	\end{enumerate}
Despues hace lo mismo para los otros procesos.

\item[\textbf{3.9}]\emph{ Incluyendo al proceso padre inicial, ?`cuantos procesos son creados por el programa?} \\
\begin{itemize}
	\item Figura 3.28: ?`Cuantos procesos son creados?\\
\#include \<stdio.h>\\
\#include \<unistd.h>\\
\\
int main()\\
\{\\
	/* fork a child process */\\
	fork();\\
	/*fork another child process */\\
	fork();\\
	/* and fork another */\\
	fork();\\
	\\
	return 0;\\
\}
\end{itemize}
R:\\
En el primer fork() copia al proceso padre y crea un hijo($H_{0}$) que sigue en la misma posicion donde fue creado y el padre queda en Wait. El proceso $H_{0}$ parte con 2 fork()'s adelante de el y luego hace un fork() creando otro hijo($H_{1}$) y $H_{0}$ queda esperando. Luego $H_{1}$ hace el fork que queda y hace otro hijo, $H_{1}$ queda en wait, que no hace nada y termina y recursivamente se hace el resto. Por lo tanto se crean 8 hijos.
\item[\textbf{3.10}]\emph{ Usando el siguiente programa, identifique los valores del PID en las lineas A,B,C y D. (Asuma que los PIDs actuales del padre y el hijo son 2600 y 2603, respectivamente)  }
\begin{itemize}
	\item Figura 3.29: ?`Cuales son los valores del PID?\\
\#include \<sys/types.h>\\
\#include \<stdio.h>\\
\#include \<unistd.h>\\
\\
int main()\\
\{\\
pid\_t pid, pid1;\\
	/* fork a child process */\\
	pid = fork();\\
	\\
	if (pidf < 0) \{ /* error ocurred */\\
		fprintf(stderr, ``Fork Failed'');\\
		return 1;\\
	\}\\
	else if (pid == 0) \{	/* child process */\\
	pid1 = getpid();\\
	printf(``child: pid = \%d'',pid); /* A */\\
	printf(``child: pid1 = \%d'',pid1); /* B */ \\
	\}\\
	else \{ /* parent process */\\
		pid1 = getpid();\\
		printf(``parent: pid = \%d'',pid); /* C */\\
		printf(``parent: pid1 = \%d'',pid1); /* D */\\
		wait(NULL);\\
	\}\\
	return 0;\\
\}
\end{itemize}
R:\\
A:0, B:2603, C:2600, D:2600
\end{itemize}
