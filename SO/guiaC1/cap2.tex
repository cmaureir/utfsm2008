\begin{itemize}
\item[\textbf{2.4}]\emph{ ?`Cu\'ales son las 3 mayores actividades de un SO con respecto a la administraci\'on de almacenamiento secundario?}\\
R:
\begin{itemize}
	\item Administraci\'on de espacio libre(Free-space management)
	\item Asignacion de almacenamiento(storage allocation)
	\item Planificacion de disco(disk scheduling)
\end{itemize}
\item[\textbf{2.7}] \emph{?`Cu\'al es el prop\'osito de los programas de sistemas?}\\
R:\\
Los programas de sistema se pueden considerar como paquetes de utiles llamadas de sistema.Ellos proveen funcionalidades basicas a los usuarios entonces esos usuarios no tienen necesidad de escribir sus propios programas para resolver problemas comunes.
\item[\textbf{2.10}]\emph{?`Por qu\'e algunos sistemas guardan el SO en un firmware, mientras otros lo hacen en el disco duro?}\\
R:\\
Para ciertos dispositivos, como PDAs y celulares, un disco con un sistema de archivos puede no estar disponible para el dispositivo.En estos casos el SO debe ser almacenado en el firmware.
\item[\textbf{2.15}]\emph{?`Cu\'ales son las 5 mayores actividades de un SO con respecto al manejo de archivos?}\\
R:
\begin{enumerate}
	\item Creacion y Eliminacion de archivos.
 	\item Creacion y Eliminacion de direcciones.
 	\item El soporte de primitivas para manipular archivos y direcciones.
 	\item El mapeo de archivos en el almacenamiento secundario
 	\item El respaldo de archivos en es medios de almacenamiento estable
\end{enumerate}

\item[\textbf{2.16}]\emph{?`Cu\'ales son las ventajas y desventajas de usar la misma interfaz de llamadas de sistema para manipular archivos y dispositivos?}\\
R:\\
\item[\textbf{2.18}]\emph{?`Cu\'ales son los 2 modelos de la comunicaci\'on ``interprocesos''??`Cu\'ales son las debilidades y fortalezas de los 2 enfoques?}\\
R:
\begin{itemize}
	\item Shared-Memory Level:
		\begin{itemize}
			\item Fortaleza:rapida, cuando los procesos estan en la misma maquina
			\item Debilidad: Procesos diferentes necesitan asegurarse de que no estan esribiendo en el mismo lugar al mismo tiempo.Los procesos necesitan abordar problemas de proteccion de memoria y sincronizacion.
		\end{itemize}
	\item Message-passing model:
		\begin{itemize}
	       		\item Fortaleza:facil de implementar que el anterior
	       		\item Debilidad:lento, porque se considera el tiempo de la configuracion de la conexion
	       	\end{itemize}
\end{itemize}

\item[\textbf{2.20}]\emph{A veces es difícil lograr un enfoque por capas si dos componentes del SO son dependientes el uno del otro. 
Identificar un escenario en el cual no est\'a claro como lograr el enfoque de capas de dos componentes del sistema que estan estrechamente unidos a sus funcionalidades.}\\
\textcolor{red}{R:}
Cuando se quiere utilizar un servicio, s\'olo puede utilizar los servicios de niveles inferiores.\\
Entonces si tenemos un servicio de un nivel que necesite alguna llamada extraordinaria de un nivel superior y el resto de las llamadas sean a niveles inferiores, estaria mal hecho el enfoque por capas, porque NO se puede dejar que se necesiten los niveles superiores.un ejemplo concreto...nose..

\item[\textbf{2.21}]\emph{?`Cu\'al es la ventaja principal de un microkernel enfocado al diseño de sistemas?}\\
R:\\
Los beneficios normalmente incluyen los siguientes:
\begin{itemize}
	\item a\~nadir un nuevo servicio no requiere modificaci\'on del kernel
	\item es mas seguro hacer operaciones en modo usuario que en modo kernel
	\item un dise\~no simple de kernel y resultados funcionales tipicosen un so mas fiable.
\end{itemize}

\emph{?`C\'omo interactuan los programas de usuarios y servicios de sistema en la arquitectura microkernel?}\\
R:\\
Se comunican mediante \emph{paso de mensajes}, el programa de usuario y el servicio nunca interactuan directamente, se comunican de manera indirecta intercambiando mensajes con el microkernel.

\emph{?`Cu\'ales son las desventajas de usar el enfoque microkernel?}\\
R:\\
Sus principales dificultades son:
\begin{itemize}
	\item la complejidad en la sincronizaci\'on de todos los m\'odulos que componen el microkernel y su acceso a la memoria
	\item mas sw de interfaz es necesario, existe la posibilidad de una perdida de rendimiento
	\item Mensajes de bugs pueden ser mas dificil de arreglar debido al largo viaje que tienen que tomar en comparacion a un kernel mononucleo.
    	\item Proceso de gestion en general puede ser muy complicado.
\end{itemize}
\item[\textbf{2.22}]\emph{?`En qu\'e formas es el enfoque de kernel modular similar al ``layered approach''(enfoque de capas)?}\\
R:\\
El enfoque de kernel modular requiere que el subsistema interactue
unos con otros a traves de interfaces cuidadosamente construidas que
se suelen reducir (en terminos de la funcionalidad que esta expuesta
a modulos externos). El Enfoque de kernel de capas es similar en este
aspecto.
\emph{?`En cu\'antas formas es diferente?}\\
R:\\
El enfoque de kernel de capas impone un orden estricto de subsistemas
de tal forma que en que en los subsistemas de capas inferiores no estan
autorizados para invocar operaciones correspondientes a capas mas altas
en otros subsistemas.
No hay restricciones en el enfoque de kernel modular, aqui los modulos
son libres de invocar operaciones unos con otros.
\end{itemize}
