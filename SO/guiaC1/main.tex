\documentclass[11pt,letterpaper]{article}
\usepackage[utf8]{inputenc} %Codificacion del texto (ISO Latin1 encoding)

\usepackage{fancyhdr} %Permite acomodar a tu gusto la parte de arriba y
% abajo del documento
\usepackage[spanish]{babel} %Permite definir el idioma del dcumento
\usepackage{graphicx} %Permite exportar imagenes en formato eps
\usepackage{url} %Tipo de fuente para correos y paginas
\usepackage{pgf}
\usepackage{fleqn}
\usepackage{amssymb}
\usepackage{fancyvrb}
\usepackage{sectsty}
\usepackage{makeidx}
\usepackage{colortbl} %Permite colocar colores a las tablas
\usepackage{booktabs}
\definecolor{red}{rgb}{1,0,0}
%%%%%%%%%%
%Margenes%
%%%%%%%%%%
\parskip 1mm %Espacio entre parrafos

\setlength{\topmargin}{0pt}

\oddsidemargin	0.5cm  % Ancho Letter 21,59cm
\evensidemargin 0.5cm  % Alto  Letter 27,81cm
\textwidth	15.5cm
\textheight	21.0cm
\headsep	4 mm
\parindent	0.5cm
%%%%%%%%%%%%%%%%%%%%%%
%Estilo del documento%
%%%%%%%%%%%%%%%%%%%%%%
\pagestyle{fancyplain}

%%%%%%%%%%%%%%%%%%%%%%%%%%%%%%%%%%%%%%%%%%%
%Fancyheadings. Top y Bottom del documento%
%%%%%%%%%%%%%%%%%%%%%%%%%%%%%%%%%%%%%%%%%%%
% Recuerde que en este documento la portada del documento no posee
% numeracion, pero de igual manera llamaremos a esa primera pagina la numero
% 1, y la que viene la dos. Esto es para tener una idea de las que
% llamaremos pares e impares
\lhead{Sistemas Operativos} %Parte superior izquierda
\rhead{\bf \it Guia C1} %Parte superior derecha
\lfoot{\it Segundo Semestre 2008} %Parte inferior izquierda. \thepage indica
% el numero de pagina
\cfoot{} %Parte inferior central
\rfoot{\bf \thepage} %Parte inferior derecha
\renewcommand{\footrulewidth}{0.4pt} %Linea de separacion inferior

% Challa

\newtheorem{theorem}{Theorem}
\newtheorem{acknowledgement}[theorem]{Acknowledgement}
\newtheorem{algorithm}[theorem]{Algorithm}
\newtheorem{axiom}[theorem]{Axiom}
\newtheorem{case}[theorem]{Case}
\newtheorem{claim}[theorem]{Claim}
\newtheorem{conclusion}[theorem]{Conclusion}
\newtheorem{condition}[theorem]{Condition}
\newtheorem{conjecture}[theorem]{Conjecture}
\newtheorem{corollary}[theorem]{Corollary}
\newtheorem{criterion}[theorem]{Criterion}
\newtheorem{definition}[theorem]{Definition}
\newtheorem{example}[theorem]{Example}
\newtheorem{exercise}[theorem]{Exercise}
\newtheorem{lemma}[theorem]{Lemma}
\newtheorem{notation}[theorem]{Notation}
\newtheorem{problem}[theorem]{Problem}
\newtheorem{proposition}[theorem]{Proposition}
\newtheorem{remark}[theorem]{Remark}
\newtheorem{solution}[theorem]{Solution}
\newtheorem{summary}[theorem]{Summary}
\newenvironment{proof}[1][Proof]{\noindent\textbf{#1.} }{\ \rule{0.5em}{0.5em}}

\newcommand{\primaria}[1]{
	\textbf{\underline{#1}}
}

\newcommand{\foranea}[1]{
	\textbf{\textsl{#1}}
}

\newcommand{\primyfor}[1]{
	\underline{\foranea{#1}}
}

\makeatletter
\newcommand\subsubsubsection{\@startsection {paragraph}{1}{\z@}%
                                   {-3.5ex \@plus -1ex \@minus -.2ex}%
                                   {1.5ex \@plus.2ex}%
                                   {\normalfont\bfseries}}
\newcommand\subsubsubsubsection{\@startsection {subparagraph}{1}{\z@}%
                                   {-3.5ex \@plus -1ex \@minus -.2ex}%
                                   {1.5ex \@plus.2ex}%
                                   {\normalfont\bfseries}}


\makeatother

%\makeindex
%%%%%%%%%%%%%%%%%%%%%%%%%%%%%%%%%%%%%%%%%%%%%%%%%%%%%%%%%%%%%%%%%%%
%%%%%%%%%%%%%%%%%%%% Aqui empieza el documento %%%%%%%%%%%%%%%%%%%%
%%%%%%%%%%%%%%%%%%%%%%%%%%%%%%%%%%%%%%%%%%%%%%%%%%%%%%%%%%%%%%%%%%%

\begin{document}

%%%%%%%%%%%%%%%%%%%%%%%%%%
%Definicion de la portada%
%%%%%%%%%%%%%%%%%%%%%%%%%%
\begin{titlepage}
    \begin{center}
	\begin{tabular}{ccc}
	    %\epsfig{file=escudo-utfsm.eps, height=1.6cm}
	      \includegraphics[height=1.6cm]{images/logoUTFSM}
	    & %Escudo de la Universidad Santa Maria
	    \hspace{0.2cm}
	    \begin{tabular}{c}
		Universidad Técnica Federico Santa María \\ \hline
		\hspace{8.0cm}
		\vspace{1.2cm}
	    \end{tabular}
	    \hspace{0.2cm}
	    &
	   % \epsfig{file=logo_DI.eps, height=1.6cm} %Logo del DI
            \includegraphics[height=1.6cm]{images/logoDI}
	\end{tabular}

	\vspace{2.5cm}
	%Titulo del Documento
	    \begin{tabular}{c}
		\Huge{\textbf{Tarea 2}}\\\\\\\\\\\\\\
		\LARGE{\textbf{``Sistema de Gesti\'on de Clientes VIP''}}\\\\
		\LARGE{\sc{Fundamentos de Ingenier\'ia}}\\
		\LARGE{\sc{de Software}}
	    \end{tabular}

	\vspace{2.5cm}
%	\begin{tabular}{c}
	\begin{center}
%	     \pgfimage[height=1.8cm]{images/now}
	%    \Huge{\emph{Now!}}
%	\end{tabular}
	\end{center}
	
        \vspace{1.5cm}

	%Nombre del (o los) autor(es)
	\begin{tabular}{lr}
            \begin{tabular}{c}
	        \large{Rodrigo Fern\'andez - 2673002-3}\\
		\large{\url{rfernand@inf.utfsm.cl}}
	    \end{tabular}
	&
	   \begin{tabular}{c}
         	\large{Cristi\'an Maureira - 2673030-9}\\ 
		\large{\url{cmaureir@inf.utfsm.cl}}
	   \end{tabular}
	\end{tabular}
        \vspace{3.5cm}\\
	%Fecha
		\large{\sc{Valpara\'iso, Octubre 2008.}}
    \end{center}
\end{titlepage}


\section*{Capitulo 1}
\label{sec:cap1}
	\subsection{Conceptos Fundamentales}
		\begin{itemize}
			\item Poblacion o Poblacion Objetivo: Conjunto de elementos sobre los que queremos hacer afirmaciones.
			\item Muestra: Subconjunto de la poblacion que se extrae para ser estudiado.
			\item Marco Muestral: Conjunto de elementos de la poblacion suceptible de ser muestreada.
		\end{itemize}
	\subsection{T\'ecnicas de Muestreo}
		\begin{itemize}
			\item Muestreo No-Aleatorizado (o No-probabilista): Se basa en el juicio personal del investigador. Puede generar buenas muestras pero no permite una evaluaci\'on de confianza.
			\item Muestreo Aleatorizado (o Probabilista): se controla la probabilidad de seleccionar un determinado individuo del marco muestral. Permite estudiar objetivamente la confianza de las generalizaciones hacia la poblaci\'on objetivo.
		\end{itemize}
		\textbf{?`Como recolectar los datos?}\\
		\begin{itemize}
			\item Muestreo no-Aleatorizado o no-Probabilistico
			\begin{itemize}
				\item Muestreo por conveniencia: Los elementos de la muestra se eligen por estar en el lugar o en el momento adecuado para la investigacio\'on. 
				\item Muestreo por juicio: Se selecciona de acuerdo a alguna caracter\'istica especifica del encuestado juzgada por el encuestador.
				\item Muestreo por cuota: Intenta mejorar la representatividad de la muestra separando a la poblaci\'on de acerdo a variables de control (edad, sexo, raza, ...). Luego a cada subgrupo se le asigna una cuota o proporci\'on de muestreo, t\'ipicamente \% de la poblacion.
				\item Muestreo tipo "bola de nieve": Se selecciona un grupo inicial. Luego, los nuevos encuestados se seleccionan en base a las referencias de los encuestados anteriores.
			\end{itemize}
			\item Muestreo Aleatorizado o Probabilistico
			\begin{itemize}
				\item Muestreo aleatorio simple: Cada elemento del marco muestral tiene la misma probabilidad de ser seleccionado y cada elemento se selecciona de manera independiente de los otros.
				 \begin{itemize}
					\item Con reemplazo: se pueden repetir elementos
					\item Sin reemplazo: no se pueden repetir elementos
				 \end{itemize}
				Se indeza a la poblacion y luego se elige un indice de manera aleatoria hasta completar el tama\~no deseado de la muestra.\\
				Generalmente se usa una tabla de numeros aleatorios.\\
				\item Muestreo sistematico: Se elige un elemento de partida aleatoriamente y el resto se elige en sucesi\'on hasta completar la muestra.\\
				Si n es el tama~no de la muestra y N el de la poblacion muestra se determina $ s = \frac{N}{n}$ , con piso.
				\item Muestreo estratificado: Antes de seleccionar los elementos, se agrupa la poblacion muestral en estratos de acuerdo a una variable importante. Dentro de cada estrato se puede proceder con Muestreo simple o sistematico.
				\item Muestreo clusterizado: Se divide a la poblacion en grupos lo m\'as homogeneos entre ellos y los mas heterogeneos internamente. Se seleccionan aleatoriamente los grupos a encuestar ya sea de manera simple o sistematica (Cada grupo se selecciona completamente: se toman todos sus elementos).
			\end{itemize}
		\end{itemize}
	\subsection{Tipos de Datos}
		\begin{itemize}
			\item Cuantitativos: operables aritm\'eticamente.
				\begin{itemize}
					\item Escala Intervalar: Tienen sentido las diferencias.
					\item Escala de Raz\'on: Tienen sentido los cuocientes.
					\item Discretos/Continuos.
				\end{itemize}
			\item Cualitativos
				\begin{itemize}
					\item Categoricos: Son solo nombres de referencia.
					\item Ordinales: Se pueden jerarquizar u ordenar.
				\end{itemize}
			\item Estructurados: formados por conjuntos de los anteriores. (Ej: grafos, matrices)
		\end{itemize}
	\subsection{Frecuencia}
		La frecuencia es el n\'umero de veces que un suceso se repite en la muestra. Llamaremos frecuencia relativa a la fracci\'on de veces que \'este aparece en la muestra.
	\subsection{Presentacion de los Datos}
		\begin{itemize}
			\item Datos Categ\'oricos: Usualmente se presenta la frecuencia con la que ocurre cada uno de los valores.(Ej: Diagramas de torta, de barras, ...)
			\item Datos Ordinales: Los diagramas de barras se suelen ordenar de acuerdo a la jerarqu\'ia natural de los valores posibles. (Ej: estratos economicos)
			\item Datos Cualitativos: Cuando son muchos es posible agruparlos en subconjuntos, pero generados en general por criterios no-estadisticos.
			\item Datos Cuantitativos: Tabligrama - El ultimo digito se expresa separado de los mas significativos.\\ \\
			\includegraphics[scale=0.3]{images/tabligrama}
			\\ \\Tablas de frecuencias - Agrupar los valores en intervalos y registrar la frecuencia de ese grupo de valores en la muestra.\\
			Histograma - Los intervalos son todos del misma tama\~no y cubren uniformemente el rango de los datos.\\
			\begin{itemize}
				\item Rango = maximo - minimo
				\item Amplitud de cada clase: A = (Rango + 1)/K
				\item k-esimo intervalo: $[a_k,b_k] = [b_{k-1},b_{k-1}+A]  $
			\end{itemize}
		\end{itemize}


\section*{Capitulo 2}
\label{sec:cap2}
\subsection{Estad\'istica Descriptiva}
Obtener informaci\'on desde una muestra, que permita entender o formular hip\'otesis acerca del fen\'omeno que se estudia por medio de:\\
Gr\'aficos: descripciones cualitativas de una muestra.\\
Estadisticas: descripciones cuantitativas de tendencia y variable de una muestra.\\

\subsection{Medidas de Tendencia}

\subsubsection{Moda}
Valor o clases de valores que se observa con mayor frecuencia
\subsubsection{Media}
Centro geom\'etrico del conjunto de valores observados.
$\overline{x} = \frac{1}{n} \sum_{i=1}^n x_i$
\subsubsection{Mediana} Valor que divide al conjunto de valores ordenados, en dos mitades. Obtiene valores mas representativos que la media, ya que no es tan sensible a valores muy distintos al resto.

\subsubsection{Percentiles}
Valores $\textbf{ordenados}$ que acumulan una cierta frecuenta relativa. El i-\'esimo percentil es el \textbf{primer} valor que acumula al menos $\frac{i}{100}$.

\subsubsection{Cuartiles}
Cuartiles $Q_1$ $\ldots$ $Q_4$ corresponden a los percentiles $\frac{25}{100}$ $\ldots$ $\frac{100}{100}$

\subsection{Medidas de Tendencia Datos Agrupados}
Permite reducir efectos de ruido o errores. Se pesa un intervalo y su frecuencia, no la frecuencia de un s\'olo valor.

\subsubsection{Media Muestral}

$\overline{x} = \sum_{i=1}^k f_i C_i$\\
donde:\\ \\
$f_i$: $\textbf{Frecuencia relativa}$ de la clase.\\
$C_i$: Marca de clase: $\frac{maximo - minimo}{2}$

\subsubsection{Moda}
Clase con \textbf{mayor frecuencia}

Moda $= L + a_M(\frac{D_1}{D_1 + D_2})$\\
donde:\\ \\
L: Limite inferior clase modal\\
$a_M$: Amplitud Clase Modal\\
$D_1$: $n_M-n_1$\\
$D_2$: $n_M-n_2$\\
$n_M$: Frecuencia absoluta Clase Modal\\
$n_1$: Frecuencia absoluta clase anterior\\
$n_2$: Frecuencia absoluta clase posterior\\

\subsubsection{Mediana}

$M_e = L + a_e \frac{\frac{1}{2}-F_{e-1}}{f_e}$\\ \\

L: Limite inferior de Clase Mediana.\\
$F_{e-1}$: \textbf{Frecuencia Relativa Acumulada} hasta antes de Clase Mediana.\\
$f_e$: frecuencia Relativa Clase Mediana.\\
$a_e$: Amplitud Clase Mediana.\\

\subsubsection{Percentiles}

$P_i = L + a_{p_{i}} \frac{\frac{i}{100} - F_{p_i-1}}{f_{p_i}}$\\ \\
L: Limite inferior percentil i-esimo.\\
$F_{p_{i-1}}$: Frecuencia Relativa acumulada antes de la clase percentil i-esimo.\\
$a_{p_i}$: Amplitud percentil i-esimo.\\
$f_{p_i}$: Frecuencia Relativa de la clase del percentil i-esimo.\\

\subsubsection{Cuartiles}

$Q_i = L + a_{C_i} \frac{\frac{i}{4} - F_{C_i-1}}{f_{C_i}}$\\ \\
L: Limite inferior cuartil i-esimo.\\
$F_{p_{i-1}}$: Frecuencia Relativa acumulada hasta antes de la clase del cuartil i-esimo.\\
$a_{p_i}$: Amplitud cuartil i-esimo.\\
$f_{p_i}$: Frecuencia Relativa de la clase del cuartil i-esimo.\\

\subsection{Medidas de Dispersi\'on}

Grado de variabilidad con respecto a las tendencias.

\subsubsection{Indice de Variacion}
Frecuencia con que no se observa la moda o la c lase modal en la muestra.\\
$T = 1 - f_m$

\subsubsection{Varianza Muestral}
Promedio de las diferencias al cuadrado con respecto a la media.\\
Para datos no agrupados:\\
$s^2 = \frac{1}{n} \sum_{i=1}^n (x_i - \overline{x})^2$\\
Para datos agrupados:\\
$s^2 = \sum_{i=1}^n f_i(x_i - \overline{x})^2$\\ \ \ \ \ donde $f_i$: Frecuencia relativa de Clase i.

\subsubsection{Desviacion Estandar}
Tiene las mismas unidades de medida que las observaciones de la muestra.\\
Para datos no agrupados:\\
$s = \sqrt{\frac{1}{n} \sum_{i=1}^n (x_i - \overline{x})^2}$\\
Para datos agrupados:\\
$s = \sqrt{\sum_{i=1}^n (x_i - \overline{x})^2}$

\subsubsection{Desviacion Media}
Promedio de las diferencias absolutas con respecto a la media, tiene las mismas unidades de medida que las observaciones de la muestra.\\
Para datos no agrupados:\\
$MD = \frac{1}{n} \sum_{i=1}^n |x_i - \overline{x}|$\\
Para datos agrupados:\\
$MD = \sum_{i=1}^n f_i|x_i - \overline{x}|$

\subsubsection{Rango}
Dferencia entre el maximo y el minimo valor observado en la muestra.

\subsubsection{Rango Percentil}
Diferencia entre $P_{90}$ y $P_{10}$. Aproximacion mas robusta al rango.\\
$PR = P_{90} - P_{10}$

\subsubsection{Rango Intercuartilico}
Distancia promedio de los cuartiles con respecto a la mediana (segundo cuartil).\\
$IQR = \frac{Q_3 - Q_1}{2}$

\subsubsection{BoxPlots}

\includegraphics[scale=0.5]{images/boxplot}\\
\textbf{OJO:} el dibujo esta malo, Q1 esta al principio de la cajita, Q2 esta en la posicion de la mediana y Q3 esta al final de la caja.
Se entiende tambien que el IRQ de la izquierda es ``-3IRQ'' y no 3IRQ
Representacion visual para describir simultaneamente, varias caracteristicas importantes tales como:
\begin{itemize}
\item Centro
\item Dispersion
\item Asimetria de la distribucion
\item Identificar valores atipicos
\end{itemize}


\section*{Capitulo 3}
\label{sec:cap3}
\begin{itemize}
\item[\textbf{3.6}]\emph{ Describa las diferencias entre short-term, medium-term y long-term scheduling}\\
R:
	\begin{itemize}
		\item \textbf{Long Term}: Determina que programas son admitidos para ejecucion y cuales deberan ser salidos. Es invocado pocas veces, segundos o minutos.
		\item \textbf{Medium Term}: Determina que procesos deben ser suspendidos o resumidos, es el que activa el context switch.
		\item \textbf{Short Term}: Determina que procesos estan listos para ocupar CPU y por cuanto tiempo. Invocado frecuentemente, milisegundos.
	\end{itemize}

\item[\textbf{3.7}]\emph{ Describa las acciones realizadas por el kernel para realizar el context-switch entre procesos }\\
R:
	\begin{enumerate}
		\item Proceso $P_{0}$ es interrumpido por llamada de sistema o interrupcion
		\item Salva el estado de $P_{0}$ en $PCB_{0}$
		\item Recupera informacion de proceso $P_{1}$ del $PCB_{1}$
		\item Ejecuta $P_{1}$ desde el estado $PCB_{1}$
	\end{enumerate}
Despues hace lo mismo para los otros procesos.

\item[\textbf{3.9}]\emph{ Incluyendo al proceso padre inicial, ?`cuantos procesos son creados por el programa?} \\
\begin{itemize}
	\item Figura 3.28: ?`Cuantos procesos son creados?\\
\#include \<stdio.h>\\
\#include \<unistd.h>\\
\\
int main()\\
\{\\
	/* fork a child process */\\
	fork();\\
	/*fork another child process */\\
	fork();\\
	/* and fork another */\\
	fork();\\
	\\
	return 0;\\
\}
\end{itemize}
R:\\
En el primer fork() copia al proceso padre y crea un hijo($H_{0}$) que sigue en la misma posicion donde fue creado y el padre queda en Wait. El proceso $H_{0}$ parte con 2 fork()'s adelante de el y luego hace un fork() creando otro hijo($H_{1}$) y $H_{0}$ queda esperando. Luego $H_{1}$ hace el fork que queda y hace otro hijo, $H_{1}$ queda en wait, que no hace nada y termina y recursivamente se hace el resto. Por lo tanto se crean 8 hijos.
\item[\textbf{3.10}]\emph{ Usando el siguiente programa, identifique los valores del PID en las lineas A,B,C y D. (Asuma que los PIDs actuales del padre y el hijo son 2600 y 2603, respectivamente)  }
\begin{itemize}
	\item Figura 3.29: ?`Cuales son los valores del PID?\\
\#include \<sys/types.h>\\
\#include \<stdio.h>\\
\#include \<unistd.h>\\
\\
int main()\\
\{\\
pid\_t pid, pid1;\\
	/* fork a child process */\\
	pid = fork();\\
	\\
	if (pidf < 0) \{ /* error ocurred */\\
		fprintf(stderr, ``Fork Failed'');\\
		return 1;\\
	\}\\
	else if (pid == 0) \{	/* child process */\\
	pid1 = getpid();\\
	printf(``child: pid = \%d'',pid); /* A */\\
	printf(``child: pid1 = \%d'',pid1); /* B */ \\
	\}\\
	else \{ /* parent process */\\
		pid1 = getpid();\\
		printf(``parent: pid = \%d'',pid); /* C */\\
		printf(``parent: pid1 = \%d'',pid1); /* D */\\
		wait(NULL);\\
	\}\\
	return 0;\\
\}
\end{itemize}
R:\\
A:0, B:2603, C:2600, D:2600
\end{itemize}


\section*{Capitulo 4}
\label{sec:cap4}
\begin{itemize}
\item[\textbf{4.1}]\emph{ Provea dos ejemplos de programaci\'on en donde multi-hebras provee un mejor rendimiento que una solucion de una hebra}\\
R:
	\begin{enumerate}	
		\item Un servidor Web que procese cada petici\'on en threads separados.
		\item Un programa GUI interactivo, por ejemplo, un debugger, donde un thread es usado para monitorear inputs, otro thread representa las aplicaciones corriendo y un tercer thread monitorea el rendimiento.
	\end{enumerate}
\item[\textbf{4.3}]\emph{ Describa las acciones realizadas por el kernel en un context-switch entre hebras a nivel del kernel }\\
R:\\
El context-switching entre threads del kernel requiere generalmente guardar los valores de los registros de la CPU desde el thread que esta siendo 'switcheado' y recuperando los registros del CPU del nuevo thread que viene en el schedule.
\item[\textbf{4.7}]\emph{ Provea dos ejemplos de programaci\'on en el cual multi-hebras no proveen un mejor rendimiento que una solucion de una hebra}\\
R:
	\begin{enumerate}
		\item Ejemplo 1:\\\\
string filenames[100];\\
/* Se crean threads del 1 al 100 */\\
create\_threads();\\
/* Cada thread crea un archivo en particular, cuyo nombre se pasa al programa ejecutado por el thread */\\
\\while (1)\{ \\
not\_created = FALSE;\\
for i = 0 .. 99 \{ \\
if (not(exists(filename[i])) \{ not\_created = TRUE; \} \\
\} \\
if (not\_created == FALSE)\{ \\
printf("Done!");\\
return;\\
\} \\
\} \\ \\

Este es el seudoc\'odigo de un programa multihebra que no desempe~na mejor rendimiento que un programa de una hebra para solucionar el mismo problema.\\
		\item Ejemplo 2:\\ \\
			Si el tiempo de computaci\'on de los procesos no difiere significativamente, un programa multihebra no es m\'as \'util en comparaci\'on a una soluci\'on de una hebra.\\
	\end{enumerate}
\item[\textbf{4.8}]\emph{ Describa las acciones realizadas por una biblioteca de hebras en un context-switch entre hebras a nivel de usuario }\\
R:\\
Para realizar el context-switch entre hebras a nivel de usuario la biblioteca de threads guarda el contexto del anterior thread en su TCB (Thread Control Block) y carga el contexto del nuevo thread. El TCB contiene un stack de punteros, un Program Counter, valores de los registros y el estado actual del thread.\\
\item[\textbf{4.10}]\emph{ Cuales de los siguientes componenes del estado del programa son compartidos por las hebras en un procesamiento multi-hebra}
	\begin{enumerate}
		\item Valores de Registro
		\item Memoria Heap
		\item Variables Globales
		\item Memoria de Stack
	\end{enumerate}
	R:
	 \begin{itemize}
		\item Heap memory
		\item Global variables
	 \end{itemize}
\item[\textbf{4.11}]\emph{ ?`Puede una soluci\'on multi-hebra usando multiples hebras de niveles de usuario obtener un mejor rendimiento en un sistema con multiprocesador que dentro de uno de 1 solo procesador? Explique.}\\
R:\\
No hay mejor rendimiento, pues el sistema operativo v\'e solo un proceso individual y no va a gestionar los diferentes threads del proceso en procesadores diferentes.\\
\item[\textbf{4.13}]\emph{ El programa muestro a continuaci\'on usa la API Pthreads. ?`Que es lo seria el output del programa en la linea C y linea P?}
	\begin{itemize}
	\item Figura 4.14: Programa en C para el Ejercicio 4.13\\
\#include <stdio.h>\\
\#include <pthread.h>\\
\\
int value = 0;\\
void *runner(void *param); /* the thread */\\
\\
int main(int argc, char *argv[])\\
\{\\
	int pid;\\
	pthread\_t tid;\\
	pthread\_attr\_t attr;\\
	\\
	pid = fork();\\
	\\
	if (pid == 0) \{ /* child process */\\
		pthread\_ttr\_init(\&attr);\\
		pthread\_create(\&tid,\&attr,runner,NULL);\\
		pthread\_join(tid,NULL);\\
		fprintf(``CHILD: valuea = \%d'',value); /* LINE C */\\
	\}\\
	else if (pid > 0) \{ /* parent process */\\
		wait(NULL);\\
		fprintf(``PARENT: valuea = \%d'',value); /* LINE P */\\
	\}\\
	\\
	void *runner(void *param) \{\\
		value = 5;\\
		pthread\_exit(0);\\
	\}\\
\}
	\end{itemize}
R:\\
Linea C, 5.\\
Linea P, 0.\\

\end{itemize}


\section*{Capitulo 5}
\label{sec:cap5}
\subsection{Probabilidad: Axiomas y Modelos}
Metodos para recolectar y analizar datos acerca de un fenomeno acerca del cual se tiene \textbf{intertudumbre}. Obtener conclusiones:
\begin{itemize}
\item Entender un fenomeno
\item Tomar decisiones
\item Controlar un fenomeno
\end{itemize}

\subsection{Metodo estadistico}
\begin{itemize}
\item Recolectar datos
\item Analizar los datos
\item Modelar el fenomeno
\item Sacar conclusiones
\end{itemize}

\subsubsection{Probabilidades}
Modelo matematico para la incertidumbre. Nocion Frecuentista generaliza la idea de frecuencia de un suceso o resultado

\subsubsection{Espacio Muestral $\Omega$}
Conjunto de resultados elementales posibles.\\
Ej. Tirar un dado dos veces.\\
$\Omega = {(1,1) (1,2) (1,3) \ldots (6,6)}$\\
Ej. Tiempo de espera en una cola de supermercado.\\
$\Omega = [0,100]$\\
\subsubsection{Eventos}
Cualquier subconjunto del espacio muestral $\Omega$ se denomina \textbf{evento}. Se busca hablar de la \textbf{probabilidad de eventos}
\subsubsection{Axiomas}
Una medida de probabilidad es una medida de la certeza de un evento. La probabilidad de un evento que debiera reflejar la certeza con que obtendremos uno de los resultados del evento.
\begin{itemize}
\item Axioma 1: $P(\Omega) = 1$
\item Axioma 2: $P(A) >= 0$
\item Axioma 3: Si A y B son eventos disjuntos P(A U B) = P(A) + P(B)
\end{itemize}
\subsubsection{Implicancia de los Axiomas}
\begin{itemize}
\item $P(A^c) = 1 - P(A)$
\item Si A $\subset$ B $\rightarrow P(A) <= P(B)$
\item P(B-A) = P(B) - P(A $\bigcap$ B)
\item P(A $\bigcup$ B) = P(B) + P(B) - P(A $\bigcap$ B)
\item $P(\bigcup A_i) <= \sum P(A_i)$
\end{itemize}

\subsubsection{Sigma Algebra}
Coleccion de eventos que son posible \textbf{medir}. Debe satisfacer propiedades minimas de cerradura para que sea util. Como se trata de subconjuntos, tiene sentido hablar de operaciones como complementos, intersecciones, uniones, diferencias.\\
Dado un espacio muestral $\Omega$, una sigma algebra es una coleccion C de sobconjuntos de $\Omega$ tal que:
\begin{itemize}
\item C $\neq \Phi$
\item Si A $\epsilon$ C entonces $(\Omega - A) \epsilon C$
\item Si $A_1 \ldots A_n \epsilon C => A_1 U \ldots U A_ni\ \epsilon\ C$ 
\end{itemize}

\subsubsection{Medida de Probabilidad}
Dado un conjunto $\Omega$ y una sigma algebra C, una medida de probabilidad es una funcion que:\\
$P: C \rightarrow R$

\begin{itemize}
\item $P(\Omega) = 1$
\item $P(A) >= 0$ para todo A $\epsilon$ C
\item Si $A_1 \ldots A_n \epsilon C$ son disjuntos, $P(A_1 \ldots \bigcup A_n) = P(A_1) + \ldots + P(A_n)$
\end{itemize}
$(\Omega, C, P)$: Espacio de Probabilidad\\
$(\Omega, C)$: Espacio Medible\\
C es la Familia de los Eventos Medibles\\
Podemos pensar como probabilidad la frecuencia con que se ve un resultado si observamos el fenomeno multiples veces.

\subsubsection{Nocion frecuentista}
Suponer repetir un experimento N veces y de estas, $N_a$ veces observamos un resultado contenido en el evento A. La probabilidad seria:\\ \\
$P(A) = lim_{n \rightarrow \infty} \frac{N_a}{N}$

\subsubsection{Nocion te\'orica}
Si tenemos un experimento que puede ocurrir de N formas, nuestro espacio muestral es finito $\Omega$.\\
Una sigma algebra posible es Pow($\Omega$).\\
Una medida de probabilidad \textbf{natural} es:\\ \\
$P(A) = \frac{|A|}{N} = \frac{|A|}{\Omega}$ donde $|A|$ es la cardinalidad de A.\\
$(\Omega, Pow(\Omega), P)$ es un espacio de probabilidad valido.\\ \\
Ej. Tirar un dado y que salga par:\\
$P({2,4,6}) = \frac{|{2,4,6,}|}{|{1,2,3,4,5,6,}|} = \frac{3}{6} = \frac{1}{2} = \frac{ResultadosFavorablesAlEventoA}{ResultadosPosibles}$

\subsubsubsection{Combinaciones}
Formas distintas de obtener r elementos de un lote de n:\\
$C(n,r) = \frac{n!}{n!(n-r)!}$

\subsection{Nocion Bayesiana}
``La probabilidad de que el cafe este frio es 0.8''\\
Es valido querer aclarar nuestro grado de incertidumbre. ?`Podr\'iamos darle una interpretaci\'on a la sentencia bajo la nocion frecuentista o teorica?.\\
La Nocion Bayesiana se entiende como la probabilidad con un grado subjetivo de certeza. Lo importante es desarrollar una forma de operar con estos \textbf{grados de certeza} para combinarlos y actualizarlos si se tienen nuevas observaciones. Es compatible con la teorica o frecuentista.




\section*{Capitulo 6}
\label{sec:cap6}
\textbf{Ejemplo:} ?`Cu\'al es la ``probabilidad'' de que tardemos m\'as de 30 minutos en la cola del almuerzo? si sabemos que son las
13:15 de la tarde?
\begin{center}
	\includegraphics[height=3cm]{images/cap6_ej1_1}
\end{center}
Entonces claramente es $\frac{20}{25+55+55}\ \approx\ 14.81\%$
\subsection{Probabilidad Condicional}
\begin{itemize}
	\item Sean A,B dos sucesos tal que $P(B)>0$
	\item La probabilidad de A condicionada a la ocurrencia de B es:
	$$P(A|B)\ =\ \frac{P(A\cap B)}{P(B)}$$
	\item Notemos que la idea de frecuencias coindicionales calza perfectamente en este modelo
	\item Centra el foco de atencion en el hecho que se sabe que ha ocurrido el evento B
	\item Estamos indicando que el espacio muestral de interes se ha ``reducido'' solo a
	aquellos resultados que definen la ocurrencia del evento B
	\item Entonces $P(A|B)$ ``mide'' la probabilidad relativa de A con respecto al espacio
	reducido B
\end{itemize}
\subsection{Se respetan los axiomas basicos}
\begin{itemize}
	\item $P(A|B)\ \geq\ 0$
	\item $P(\Omega|B)\ =\ 1$
	\item Sean $A_{1},A_{2},\ldots,A_{n}$ disjuntos $A_{i}\cap A_{j}\ =\ \emptyset\ \forall i\neq j$
	$$P(\cup A_{i}|B)\ =\ \sum P(A_{i}|B)$$
\end{itemize}
\textbf{Ejemplo2:} Si lanzamos dos dados (4 caras) ?`Cu\'al es la probabilidad de que el m\'aximo
	de los resultados sea par dado que el m\'inimo de los resultados es 3?\\
	$$\Omega\ =\ {(1.1),(1.2),(1.3),\ldots,(4.4)}$$
	donde la minima es 3 $B\ =\ {(3.3),(3.4),(4.3)}$\\
	donde el maximo es par $A\ =\ {(3.4),(4.3)}$\\
	$$P(A|B)\ =\ \frac{2}{3}\ \Longleftrightarrow \frac{P(A\cap B)}{P(B)}\ =\ \frac{\frac{2}{16}}{\frac{3}{16}}\ =\ \frac{2}{3}$$
\textbf{Ejemplo3:} En una encuesta se ha determinado que los fines de semana el 45\% de la poblacion
	lee la tercera, el 35\% lee el mercurio y el 5\% lee ambos diarios. ?`Cu\'al es la probabilidad
	de que un lector de la tercera lea tambi\'en el mercurio?\\
	$$\frac{P(A\cap B)}{P(B)}\ =\ \frac{\frac{5}{100}}{\frac{45}{100}}\ =\ \frac{1}{9}$$
\textbf{Ejemplo4:} En una f\'abrica se ha recopilado la siguiente informaci\'on (expresar como probabilidades)
	\begin{itemize}
		\item El 25\% de las piezas con fallas superficiales son funcionalmente defectuosas
		\item Se sabe que el 10\% de las piezas manufacturadas tienen fallas visibles
		en la superficie
		\item Tambi\'en se ha encontrado que el 5\% de las piezas que no tenian fallas
		superficiales son funcionalmente defectuosas
	\end{itemize}
	\begin{center}
		\includegraphics[height=4cm]{images/cap6_ej4_1}
	\end{center}
	B = \{ Pieza funcionalmente defectuosa \}\\
	A = \{ Pieza tiene una falla visible en la superficie \}\\
	$P(B|A)\ =\ 25\%$\\
	$P(A)\ =\ 10\%$\\
	$P(B|A^{c})\ =\ 5\%$\\
	$P(A^{c})\ =\ 90\%$\\
	$P(B)\ =\ P(B|A)P(A)\ +\ P(B|A^{c})P(A^{c})\ =\ \frac{25}{100}\cdot \frac{10}{100}\ +\ \frac{5}{100}\cdot \frac{90}{100}\ =\ \frac{7}{100}$\\
	$P(A\cap B)\ =\ P(B\cap A)\ =\ \frac{P(B|A)}{P(A)}\ =\ \frac{25}{1000}\ =\ \frac{1}{40}$\\
	$$P(A|B)\ =\ \frac{P(A\cap B)}{P(B)}\ =\ \frac{\frac{1}{40}}{\frac{7}{100}}\ =\ \frac{5}{14}$$
\subsection{Distintos Casos de la Probabilidad Condicional}
	\begin{center}
		\includegraphics[height=4cm]{images/cap6_1}\\
		\includegraphics[height=4cm]{images/cap6_2}\\
		\includegraphics[height=4cm]{images/cap6_3}\\
		\includegraphics[height=4cm]{images/cap6_4}
	\end{center}
\subsection{Probabilidad Marginal}
	Si estudiamos la relación entre una serie de eventos $A,B,C$, llamaremos ``probabilidades
	marginales'' a las probabilidades no condicionales $P(A),\ P(B)\ y\ P(C)$. 
\subsection{Regla de Bayes}
	\begin{itemize}
		\item Sean A, B dos sucesos tal que $P(A)$, $P(B)>0$.
		\item Establece una relacion entre las probabilidades condicionales $P(A|B)$ y $P(B|A)$
		$$P(A|B)\ =\ \frac{P(B|A)P(A)}{P(B)}$$
		\item Se sigue inmediatamente de la definicion de probabilidad condicional
	\end{itemize}
\subsection{Probabilidad Total}
	Sean B1, B2,....,Bn  eventos mutuamente excluyentes tal que su uni\'on conforma
	el espacio muestral:
	$$P(\bigcup_{i=1}^{n}B_{i})\ =\ 1$$
	Entonces:
	$$P(A)\ =\ P(A|B_{1})P(B_{1})\ +\ \ldots\ +\ P(A|B_{n})P(B_{n})$$
	\begin{center}
		\includegraphics[height=4cm]{images/cap6_5}\\
	\end{center}
	\textbf{Ejemplo5:} Un producto se fabrica en 5 plantas que producen el 20\%, 25\%, 30\%,
	 15\% y 10\% respectivamente. Las probabilidades de fallas en cada planta est\'an dadas
	por: 0.2, 0.1, 0.15, 0.3, 0.0 ?`Cu\'al es la probabilidad de que un producto venga fallado?\\
	$P(B_1)\ =\ 20\%$\\
	$P(B_2)\ =\ 25\%$\\
	$P(B_3)\ =\ 30\%$\\
	$P(B_4)\ =\ 15\%$\\
	$P(B_5)\ =\ 10\%$\\
	$P(A)\ =\ P(A|B_{1})P(B_1)\ +\ \ldots\ +\ P(A|B_5)P(B_5)$\\
	$$P(A)\ =\ 0.2\cdot \frac{20}{100}\ +\ \ldots\ +\ 0.0\cdot \frac{10}{100}\ =\ \frac{15.5}{100}\ =\ 15.5\%$$
	\textbf{Ejemplo6:} Continuando con el anterior...supongamos de que se elige aleatoriamente
	un producto y se encuentra que est\'a fallado. ?`Cu\'al es la probabilidad que sea
	manufacturado en Planta $B_3$?\\
	$$P(B_{3}|A)\ =\ \frac{P(A|B_{3})P(B_{3})}{P(A)}\ =\ \frac{0.15\cdot \frac{30}{100}}{0.155}\ =\ 0.29\ =\ 29\%$$
\subsection{Independencia}
	\begin{itemize}
		\item Dos eventos A y B se dicen independientes ssi:
		$$P(A\cap B)\ =\ P(A)P(B) \rightarrow\ P(A|B)\ =\ P(A)\ y\ P(B|A)\ =\ P(B)$$
		\item Sean ${A_{i}:i\epsilon I={1,2,3,\ldots,k}}$ una colecci\'on de eventos de
		 $(\Omega,\xi,P)$. Se dice que los elementos son conjuntamente independientes 
		 para todo subconjunto de indices $J$:
		$$P(\bigcap_{j\epsilon J}A_{j})\ =\ \prod_{j\epsilon J}P(A_{i})$$
	\end{itemize}
	\textbf{Ejemplo7:} Sea $(\Omega,2^{\Omega},P)$ modelo de probabilidad.
		$$\Omega\ =\ {(1,0,0),(0,1,0),(0,0,1),(1,1,1)}$$
		$$P({w_{i}})\ =\ \frac{1}{4}$$
		Sean $A_{1},\ A_{2},\ A_{3}$ eventos de $(\Omega,2^{\Omega},P)$:\\
		\begin{itemize}
			\item $A_{1}:$ Primera coordenada es 1
			\item $A_{2}:$ Segunda coordenada es 1
			\item $A_{3}:$ Tercera coordenada es 1
		\end{itemize}
		$P(A_{1})\ =\ \frac{1}{2}$\\
		$P(A_{2})\ =\ \frac{1}{2}$\\
		$P(A_{3})\ =\ \frac{1}{2}$\\
		$P(A_{1}\cap A_{2})\ =\ \frac{1}{4}\ =\ P(A_{1})P(A_{2})$\\			
		$P(A_{1}\cap A_{3})\ =\ \frac{1}{4}\ =\ P(A_{1})P(A_{3})$\\			
		$P(A_{2}\cap A_{3})\ =\ \frac{1}{4}\ =\ P(A_{2})P(A_{3})$\\
		Todos son independientes (Independencia de a Pares)\\
		$P(A_{1}\cap A_{2}\cap A_{3})\ =\ \frac{1}{4}\ \neq\ P(A_{1})P(A_{2})P(A_{3})$\\
		No son independientes como Familia\\


\end{document}
