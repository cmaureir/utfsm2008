\documentclass[letterpaper,spanish,11pt]{article}
\usepackage[utf8]{inputenc}    % Agregar y acentos
\usepackage{babel}               % Soporte multilenguajes
%\usepackage{avant}               % Tipo de fuente
%\usepackage{fancyheadings}       %% Topes y pies de p�ina
%\usepackage[dvips]{graphicx}     % Inclusion de imagenes .eps
\usepackage{amsmath,amsthm,url}
%\usepackage{url}                 % Agregar Links soporte de ~
\usepackage{verbatim}
%\usepackage{geometry}
\usepackage{color}
\usepackage{amsfonts}
\usepackage{amssymb}
%\usepackage{txfonts}
%\usepackage{pxfonts}
%\usepackage{fancybox}
\usepackage{latexsym}
%\usepackage{fancyvrb}
\usepackage{graphicx}
%\usepackage{pstricks}
\usepackage{setspace} % paquete para interlineado
\usepackage{url}
\usepackage{colortbl}
\usepackage{multirow}
\usepackage{slashbox}
\usepackage{rotating}
\definecolor{rojo}{rgb}{1,0,0}
\oddsidemargin -0.1in
\topmargin -0.5in
\textwidth 6.7in
\textheight 8.5in

\begin{document}
\textbf{Guttmans}\\\\\\

\textbf{Asociaci\'on de variables en escala nominal}\\
	Supongamos que trabajaos con la siguientes variables nominales: el sexo y el hecho de fumar o no.
	Tenemos una muestra de 98 individuos con la siguiente tabla de contingencia:\\
	\begin{center}
     \begin{tabular}{ | c | c | c | c | }
        \hline
        $Sexo \diagdown Fumar$ & Si & No & Total \\
        \hline	
          Masculino & 30 & 25 & 55 \\
          Femenino  & 15 & 28 & 43 \\
          Total     & 45 & 53 & 98 \\
        \hline
     \end{tabular}
	\end{center}
	Si debieramos adivinar el sexo de un individuo de la muestra diriamos ``Masculino'' (pues es el sexo mayoritario en la muestra, la clase
	modal de la variable sexo) teniendo la posibilidad de equivocarnos en 43 de los 98 casos. Observemos si la otra variable nos entrega
	informaci\'on respecto a la variable sexo. Supongamos que sabemos si el individio fuma o no y adivinamos suy sexo.
	Si fuma diremos que es ``Masculino'' equivoc\'andonos en 15 casos (de los 45 fumadoires) si no fuma diremos que su sexo es ``Femenino'' con un error
	de 25 casos. Conociendo si fuma o no , el error de adivinar el sexo es ahora (15 + 25) casos , con lo cual hemos podido reducir el error total
	de 43 a 40 casos.\\\\
	Podemos definir un coeficiente de asociacion entre ambas variables consistentes en el cuociente de asociacion entre ambas variables
	consistentes en el cuociente entre la reduccion del error y el error total primitivo.\\\\
	La variable fumar en este ejemplo hace el rol de \textbf{variable\ predictora} y la variable ``sexo'' el rol de \textbf{variable\ dependiente}.\\\\
	Llamaremos \textbf{coeficiente\ de\ preficcion\ de\ Guttmans} a:
	$$\lambda\ =\ \frac{reduccion\ en\ el\ error}{error\ original\ total}$$
	ejemplo:\\
	$$\lambda\ =\ \frac{43-40}{43}\ =\ \frac{3}{43}$$
	OBS.:
	\begin{enumerate}
		\item \textbf{0\ $<$\ $\lambda$\ $<$\ 1} Si $\lambda$ se acerca a cero, la variable predictora reduce muy poco el error en la variable dependiente y su asociacion entre ambas es peque\~na.
		\item Si las variables intercambian su rol es probable que la nueva variable predictora provoque una fuerte reduccion en el error de la nueva variable dependiente
	\end{enumerate}
	En el ejemplo si intercambiamos el rol de las variables de modo que el ``sexo'' sea variable predictora y ``fumar o no'' sea la variable dependiente se tiene:\\
	\begin{itemize}
		\item Error original: 45 casos
		\item Error conociendo la variable ``sexo''
		\item Si el sexo es masculino: 25 casos
		\item Si el sexo es Femenino: 15 casos
		\item Total: 40 casos
	\end{itemize}
	Entonces:\\
	$$\lambda\ =\ \frac{45-40}{45}\ =\ \frac{5}{45}\ =\ \frac{1}{9}$$
	Algo mejor que cuando las variables desarrollaban suy rol primitivo\\
	El coeficiente de predicci\'on de \textbf{Guttmans} se puede generalizar a tablas de contingencia mayores de 2 x 2 a trav\'es de la siguiente f\'ormula:\\
	$$\lambda\ =\ \frac{\sum{n_{i\ast}\ -\ n_{\ast d}}}{n\ -\ n_{\ast d}}$$

	\begin{itemize}
		\item \emph{$n_{i \ast}$} frecuencia absoluta maxima dentro de cada subclase de la variable predictora
		\item \emph{$n_{\ast d}$} frecuencia absoluta maxima dentro de cada subclase de la variable dependiente
		\item \emph{n} numero de individuos
	\end{itemize}
	En el ejemplo (con el rol inicial de las variables)
	$$\lambda\ =\ \frac{(30+28)-55}{98-55}\ =\ \frac{3}{43}$$
	OBS.: Para superar el problema de cual es la variable predictoria y cual es la dependiente, usualmente se utiliza
	el coeficiente de \textbf{Guttmans}haciendo a ambas variables jugar los dos roles.
	$$\lambda\ =\ \frac{Red\ en\ el\ error\ sist.\ 1\ +\ Red\ en\ el\ sist.\ 2}{error original\ sist. 1\ +\ error\ original\ sist.\ 2}$$
	En nuestro ejemplo resultar\'ia:\\
	$$\lambda\ =\ \frac{3+5}{43+45}\ =\ \frac{8}{88}\ =\ \frac{1}{11}$$

\end{document}
