\subsection{Variables Aleatorias}
	\subsubsection{\bf Tendencia de la variable.}
	\subsubsubsection{Esperanza}
	Definicion: La esperanza o valor esperado de una variable aleatoria X se define como.
	$$ E(X) = \sum_{i=1}^n x_i \cdotp p(x_i) ,\ caso\ discreto$$
	$$ E(X) = \int_{- \infty}^{\infty} x \cdotp p(x) ,\ caso\ continuo$$
	En general, el valor esperado de una funci\'on g(X) de la variable aleatoria estar\'a dado por:
	$$ E(g(X)) = \int_{- \infty}^{\infty} g(x)p(x) ,\ caso\ discreto $$
	$$ E(g(X)) = \sum_{i=1}^{n} g(x_i)p(x_i) ,\ caso\ continuo $$
	Notemos que la esperanza es un operador lineal:
	$$ E(a \cdotp g(x) + b \cdotp h(x)) = a \cdotp E(g(x)) + b \cdotp E(h(x))$$
	\subsubsubsection{Mediana}
	Definicion: La mediana de una variable aleatoria discreta X se define como el valor $ x_0.5 $ tal que
	$$ \sum_{x_i < x_{0.5}} p(x_i) < 0.5 $$
	$$ \sum_{x_i \ge x_{0.5}} p(x_i) \ge 0.5 $$
	Definicion: La mediana de una variable aleatoria continua X se define como
	$$ x_{0.5} = F^{-1}(0.5)$$
	Es decir
	$$ \int_{- \infty}^{x_{0.5}} p(x) = 0.5 $$
	\subsubsubsection{Percentiles Arbitrarios}
	Definicion: El percentil de orden q asociado a una variable aleatoria continua se define como el valor $ x_q $ que satisface
	$$ \sum_{x_i < x_q} p(x_i) < q  $$
	$$ \sum_{x_i \ge x_q} p(x_i) \ge q $$
	Con $ 0 \le q \le 1 $ .\\
	Definicion: El percentil de orden q asociado a una variable aleatoria continua se define como
	$$ x_q = F^{-1}(q) $$
	Con $ 0 \le q \le 1 $ .\\
	\subsubsubsection{Moda}
	Definicion: Un valor m se dice una moda o valor modal de una variable aleatoria discreta X si
	$$ m = arg\ max_{i}\ p(x_i)  $$
	Definicion: UN valor m se dice una moda o valor modal de una variable aleatoria continua X si
	$$ \frac{\partial f}{dx} \mid_{x=m} = 0  $$
	\subsubsection{\bf Dispersi\'on al rededor de su tendencia.}
	\subsubsubsection{Varianza}
		Definicion: La varianza de una variable aleatoria X se define como 
		$$ Var(X) = E((X - E(X))^2)  $$
		\begin{itemize}
			\item Caso Discreto
				$$ Var(X) = \sum_{i=1}^{n} (x_i - E(X))^2 p(x_i) $$
			\item Caso Continuo
				$$ Var(X) = \int_{- \infty}^{\infty} (x - E(X))^2 p(x)dx  $$
		\end{itemize}
	\subsubsubsection{Desviaci\'on Est\'andar}
		Definicion: La desviaci\'on estandar de una variable aleatoria X se define como
		$$ \sigma = \sqrt{Var(X)}  $$
	\subsubsubsection{Rango Inter-Cuart\'ilico}
		Definicion: El rango inter-cuartilico asociado a una variable aleatoria se define como
		$$ IQR = x_{3/4} - x_{3/4}  $$
		O sea, la diferencia entre percentiles 0.75 y 0.25.
	
	\subsubsection{Momentos}
		Son las medidas descriptivas que caracterizan la distribucion de probabilidad de una variable aleatoria.\\
		\\ {\bf Idea}
		\begin{itemize}
		\item Generar medidas descriptivas que caractericen la distribucion de probabilidad de una variable aleatoria.
		\item Usar potencias de la variable aleatoria, tal como los monomios de una expansion de Taylor.
		\end{itemize}
		Definicion: El k-\'esimo momento de una variable aleatoria X se define como
		$$ \mu_{k}^0 = E(X^k),\ \ \ k \epsilon N  $$
		Nota: La esperanza es el primer momento :O .
	\subsubsection{Momentos Centrales}
		Definicion: El k-\'esimo momento central de una variable aleatoria X se define como
		$$ \mu_k = E((X - E(X))^k),\ \ \ k \epsilon N  $$
		Nota: La varianza es el segundo momento central. El primer momento central es nulo :O.\\
	\subsubsection{Funcion generadora de Momentos}
		Definicion: La funcion generadora de momentso (fgm) asociada a una variable aleatoria X se define como
		$$ m_x : R \rightarrow R  $$
		$$ m_x(t) = E(exp(t \cdotp X))  $$
		\begin{itemize}
			\item Caso Discreto
			$$ m_x(t) = \sum_{i=1}^n exp(t \cdotp x_i)p(x_i)  $$
			\item Caso Continuo
			$$ m_x(t) = \int_{-\infty}^{\infty} exp(t \cdotp x)p(x)dx  $$
		\end{itemize}
		Proposicion: El l-esimo momento de la fgm de una variable aleatoria X esta dado por
		$$ \mu_{k}^0 = \frac{dm_x(t)}{dt} \mid_{t=0}  $$

