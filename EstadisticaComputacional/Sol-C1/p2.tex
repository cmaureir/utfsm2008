%enunciado
Chile, Marzo 2010. El nuevo gobierno ha lanzado un programa de salud pública consistente en la instalación
de dispensadores de condones en los baños públicos de parques, bares, restaurantes y estaciones de servicio.
Para pemitir su distribución a bajo costo (100 pesos la unidad) se han inaugurado una marca estatal (jaguares)
con 3 plantas A, B, C responsables de producir el 45\%, 30\% y 25\% del Total respectivamente.
Se ha calculado que la probabilidad de que las plantas entreguen un condón con fallas es del 5\%, 15\% y 20\%
respectivamente.

%preguntas
\begin{itemize}
	\item Si compramos 10 unidades, ¿cuál es la probabilidad de que al menos uno salga malobrado?.
	En este caso, ¿de qué planta es más probable que venga?.\\
	%respuesta
	$P(A)\ =\ 45\%$\\
	$P(B)\ =\ 30\%$\\
	$P(C)\ =\ 25\%$\\
	Sea $P(\alpha)$ la probabilidad de que salga 1 condón fallado.\\
	$P(\alpha)\ =\ P(\alpha|A)\cdot P(A)\ +\ P(\alpha|B)\cdot P(B)\ +\ P(\alpha|C)\cdot P(C)\ $\\
	$P(\alpha)\ =\ 0.05\cdot 0.45\ +\ 0.15\cdot 0.3\ +\ 0.25\cdot 0.2\ =\ 0.11\ \approx 11\% $\\
	Para saber de que planta es mas probable que venga, solo calculamos la probabilidad por cada planta:\\
	$P(A|\alpha)\ =\ \frac{P(\alpha|A)P(A)}{P(\alpha)}\ =\ \frac{0.05\cdot 0.45}{0.11}\ =\ 0.20 $\\
	$P(B|\alpha)\ =\ \frac{P(\alpha|B)P(B)}{P(\alpha)}\ =\ \frac{0.15\cdot 0.30}{0.11}\ =\ 0.40 $\\
	$P(C|\alpha)\ =\ \frac{P(\alpha|C)P(C)}{P(\alpha)}\ =\ \frac{0.25\cdot 0.20}{0.11}\ =\ 0.45 $\\
	Claramente es mas probable que venga de la Planta C.

  	\item Marzo 2011. Para mejorar la calidad, los condones ahora se empacan y somenten a pruebas
	de manera centralizada, mediante un procedimiento que ha mostrado fallas inferiores al 10\%.
	¿Con qué probabilidad un condón no es rechazado por la planta empacadora?\\
	%respuesta
	De manera centralizada significa que ya los de todas las plantas pasan por la central de la empresa.\\
	La Probabilidad de que salga un condon fallado de las plantas es 11\% entonces si el nuevo mecanismo
	arroja fallas inferiores al 10\%, entonces el 90\% de los 11\% son detectados malos:
	$$0.9\cdot 0.11\ =\ 0.099\ \approx 9.9\%$$
	Entonces no es rechazado un condon con un 90.1\%

  	\item Después de implantado este último procedimiento, ¿Cuál es la probabilidad de el ilustre
	ciudadano obtenga un condón con fallas en la máquina dispensadora?\\
	%respuesta
	solo es el 10\% del 11\% de error, por lo tanto
	$$0.1\cdot 0.11\ =\ 1.1\%$$


\end{itemize}
