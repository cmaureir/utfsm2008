\documentclass[letterpaper,spanish,11pt]{article}
\usepackage[utf8]{inputenc}    % Agregar y acentos
\usepackage{babel}               % Soporte multilenguajes
%\usepackage{avant}               % Tipo de fuente
%\usepackage{fancyheadings}       %% Topes y pies de p�ina
%\usepackage[dvips]{graphicx}     % Inclusion de imagenes .eps
\usepackage{amsmath,amsthm,url}
%\usepackage{url}                 % Agregar Links soporte de ~
\usepackage{verbatim}
%\usepackage{geometry}
\usepackage{color}
\usepackage{amsfonts}
\usepackage{amssymb}
%\usepackage{txfonts}
%\usepackage{pxfonts}
%\usepackage{fancybox}
\usepackage{latexsym}
%\usepackage{fancyvrb}
\usepackage{graphicx}
%\usepackage{pstricks}
\usepackage{setspace} % paquete para interlineado
\usepackage{url}
\usepackage{colortbl}
\usepackage{multirow}
\usepackage{slashbox}
\usepackage{rotating}
\definecolor{rojo}{rgb}{1,0,0}
\oddsidemargin -0.1in
\topmargin -0.5in
\textwidth 6.7in
\textheight 8.5in

\begin{document}

\begin{titlepage}
%\begin{figure}[htbp]
\begin{center}
\includegraphics[height=3.5cm]{images/logo_latex}
\end{center}
%\end{figure}
\vspace{1.5cm}
\begin{center}
\textbf{\Huge{Laboratorio de estad\'istica}}\\[0.2cm]
\textbf{\Huge{computacional}}\\[0.7cm]
\textbf{\huge{Informe \# 4}}\\[0.7cm]
\textbf{\huge{``Intervalos de confianza''}}\\[0.3cm] 
\today\\[1.5cm]
\end{center}
\vspace{2cm}
\begin{flushright}
\large{\textbf{Profesor de C\'atedra}} \\
\large{Ricardo \~{N}anculef} \\[0.5cm]
\large{\textbf{Ayudantes de laboratorio}}\\
\large{Milciades Reyes}\\
\large{Fernando Herrera}\\[0.5cm]
\large{\textbf{Integrantes}} \\
\large{Esteban Bombal 2673004-k} \\
\large{Rodrigo Fernandez 2673002-3} \\
\large{Cristian Maureira 2673030-9} \\
\large{Gabriel Zamora 2673070-8} \\
\end{flushright}
\end{titlepage}

\section{Descripci\'on del Fen\'omeno y del Muestreo}

\begin{enumerate}
\item A partir de su observaci\'on de la poblaci\'on, formule hip\'otesis acerca del fen\'omeno que estudia.

\begin{itemize}
	\item Hipotesis: ''Los cigarros light duran menos en ser fumados que los cigarros corrientes (filtro cafe)''
\end{itemize}

\item Determinar y describir las variables a medir.
	\begin{itemize}
		\item Tiempo de consumo de un cigarro tipo light (filtro blanco). (Continua)
		\item Tiempo de consumo de un cigarro tipo corriente (filtro caf\'e). (Continua)
		\item Numero de personas en la muestra. (Discreta)
	\end{itemize}

\item Definir y justificar una metodolog\'ia de muestreo.
	\begin{itemize}
		\item Decidimos proceder la toma de muestreo precisamente en los lugares de descansos mas concurridos de la UTFSM, ll\'amese Patio Central, Patio del Ca~n\'on, entradas del Edificio C, etc. \\
		Nuestro punto es que registrar nuestra muestra dentro de nuestra Universidad presenta varios beneficios, por ejemplo, que la misma toma de muestras resulta menos invasiva para la poblaci\'on, al encontrarnos dentro de un recinto donde la toma de muestras y los datos son comunes. \\
		Otro tema es el uso de los datos. Aparte de comprobar \'esta hipotesis, podemos estudiar varias otras hipotesis de inter\'es comunitario, como el grado de stress que presentan los alumnos frente a los cert\'amenes finales, segun el tiempo en que se demoran en fumarse un cigarro, lo cual no haremos en este momento por tratarse de un informe sobre otros m\'etodos.
	\end{itemize}

\item Recolectar los datos.

\begin{itemize}
	\item Se recolectaron 100 datos, 50 mediciones para cigarros ligth y 50 mediciones para cigarros corrientes (de filtro cafe). 
\end{itemize}


\end{enumerate}

\section{Intervalos de confianza}

\begin{enumerate}
\item Calcular el intervalo de confianza de tiempo promedio de los cigarros tipo light, con un 75 \%, 85 \% y
95 \% de confianza. (hint: varianza poblacional desconocida).

Teniendo en cuenta que:\\

$P(\overline{x} - t_{1-\frac{\alpha}{2}}\frac{\widehat{\sigma}}{\sqrt{n}} \leq \mu_x \leq \overline{x} + t_{\frac{\alpha}{2}}\frac{\widehat{\sigma}}{\sqrt{n}}) = 1 - \alpha$\\
$\overline{x} = 381.244$\\
$\widehat{\sigma} = 58.55714$\\
$n = 50$\\
t = Distribucion t-student con n-1 grados de libertad\\

Por lo que: \\

\begin{itemize}

\item $\alpha = 0.25$

$\mu_x \in$ [371.50399 , 390.98581]

\item $\alpha = 0.15$

$\mu_x \in$ [369.00691 , 393.48288]

\item $\alpha = 0.05$

$\mu_x \in$ [364.42532 , 398.06447]

\end{itemize}

\item Una m\'aquina f\'abrica focos de 75 watts, donde el tiempo de vida \'util de los focos presenta un comportamiento de distribuci\'on
normal, donde el 16\% de los focos producidos duran menos de 1000.111 horas y el 12\% dura mas de 1043.5 horas
\begin{itemize}
	\item Calcule la media y varianza de la duraci\'on de los focos.\\
		$X~N(\mu,\sigma^{2})$\\
		Podemos desarrollar un sistema de ecuaciones, por cada porcentaje.\\
		Primero con($16\%$):\\
		$$P(X<1000.111)\ =\ 0.16$$
		$$P(Z<\frac{1000.111-\mu}{\sigma})\ =\ 0.16$$
		Mediante el comando \emph{qnorm(0.16)}, llegamos a:\\
		$$-0.99=\frac{1000.111-\mu}{\sigma}$$
		Por otro lado, con el otro porcentaje($12\%$):\\
		$$P(X>1043.5)\ =\ 0.12$$
		$$P(Z>\frac{1043.5-\mu}{\sigma})\ =\ 0.12$$
		Mediante la tabla:\\
		$$1.17=\frac{1043.5-\mu}{\sigma}$$
		Resolviendo el sistema de ecuaciones\\
		$$\sigma\ =\ 20.0875$$
		$$\mu\ =\ 1019.998$$
		$therefore\ Media=\mu=1019.988\ y\ Varianza=\mu^{2}=1040395.920 $\\
	\item ?`Cu\'al es la probabilidad de que un foco dure menos de 1000 horas?\\
		$P(X<1000)\ =\ 0.1597358\ $\\ %pnorm(1000,1019.998,20.0875)
	
	\item ?`Cu\'al es la probabilidad de que un foco dure m\'as de 1045 horas?\\
		Por simetr\'ia, es lo mismo que dure menos que 1045 horas\\
		$P(X>1045)\ =\ 1\ -\ P(X<1045)\ =\ 1\ -\  0.8933706\ =\ 0.1066294$\\ %pnorm(1000,1019.998,20.0875)
	\item ?` Que proporci\'on de focos duran entre 1000 y 1030 horas?\\
		% P(X<1030)-P(X<1000)
		% pnorm(1030,1019.998,20.0875)-pnorm(1000,1019.998,20.0875)
		$P(X<1030)-P(X<1000)\ =\ 0.5309946$\\
		
	\item Calcule $P(\mu - \sigma < X < \mu + \sigma)$, $P(\mu - 2\sigma < X < \mu + 2\sigma)$ y $P(\mu - 3\sigma < X < \mu + 3\sigma)$\\
		$P(999.9105 < X < 1040.086)\ =\ P(X<1040.086)-P(X<999.9105)\ =\ 0.6826955$\\
		$P(979.8230 < X < 1060.173)\ =\ P(X<1060.173)-P(X<979.8230)\ =\ 0.9544997$\\
		$P(959.7355 < X < 1080.261)\ =\ P(X<1080.261)-P(X<959.7355)\ =\ 0.9973003$\\

	\item Calcule las probabilidades anteriores usando la tabla y compare los resultados. (hint: estandarice).

	$P(X<1040.086)-P(X<999.9105)$\\
	$ = P(\frac{X-\mu}{\sigma} < \frac{1040.086-\mu}{\sigma}) - P(\frac{X-\mu}{\sigma} < \frac{999.9105-\mu}{\sigma})$\\
	$ = P(Z < 1) - P(Z < -1)$\\
	$ = 1 - P(Z > 1) - P(Z > 1)$\\
	$ = 1 - 2 P(Z > 1)$\\
	$ = 1 - 2 * 0.1587$\\
	$ = 0.6826$\\

        $P(X<1060.173)-P(X<979.8230)$\\
        $ = P(\frac{X-\mu}{\sigma} < \frac{1060.173-\mu}{\sigma}) - P(\frac{X-\mu}{\sigma} < \frac{979.8230-\mu}{\sigma})$\\
        $ = P(Z < 2) - P(Z < -2)$\\
	$ = 1 - P(Z > 2) - P(Z > 2)$\\
	$ = 1 - 2 P(Z > 2)$\\
	$ = 1 - 2 * 0.0228$\\
	$ = 0.9544$\\

        $P(X<1080.261)-P(X<959.7355)$\\
        $ = P(\frac{X-\mu}{\sigma} < \frac{1080.261-\mu}{\sigma}) - P(\frac{X-\mu}{\sigma} < \frac{959.7355-\mu}{\sigma})$\\
        $ = P(Z < 3) - P(Z < -3)$\\
	$ = 1 - P(Z > 3) - P(Z > 3)$\\
	$ = 1 - 2 P(Z > 3)$\\
	$ = 1 - 2 * 0.0013$\\
	$ = 0.9974$\\

	\item Realice gr\'aficos de la funci\'on de densidad de probabilidad y de la funci\'on de distribuci\'on.
	% x<-seq(900,1100,10)
	\includegraphics[scale=0.5]{images/2_2-dnorm}\\	% plot(x,dnorm(x,1019.998,20.0875),type="l") % Densidad
	\includegraphics[scale=0.5]{images/2_2-pnorm}	% plot(x,pnorm(x,1019.998,20.0875),type="l") % Distribucion
	\item Var\'ie el o los valores de los par\'ametros de la distribuci\'on y comente lo observado en los gr\'aficos de la funci\'on de densidad y de distribuci\'on. (2 casos).

	\begin{itemize}
		\item Funcion de Densidad\\
	Variando Mu (900.998):\\
	\includegraphics[scale=0.5]{images/2_2-dnorm-variado1}\\
	Variando Mu (1000.998):\\
	\includegraphics[scale=0.5]{images/2_2-dnorm-variado2}\\
	Como podemos observar, al variar Mu se refleja una traslaci\'on de la funci\'on.\\
		\item Funcion de Distribucion\\
	        Variando Mu (950.998):\\
        \includegraphics[scale=0.5]{images/2_2-pnorm-variado1}\\
        Variando Mu (1050.998):\\
        \includegraphics[scale=0.5]{images/2_2-pnorm-variado2}\\
        En la funcion de distribucion, si variamos Mu podemos ver que se desplaza el inicio de la funci\'on.\\
			
	\end{itemize}
\end{itemize}

\item Calcule los intervalos de confianza del punto 1, suponiendo que la varianza poblacional es igual a la
varianza muestral ($\sigma$ = S), es decir. con varianza poblacional conocida. ?`es mucha la diferencia?

Teniendo en cuenta que: \\

$\overline{x} = 381.244$\\
$\widehat{\sigma} = 58.55714$\\
$n = 50$\\

Por lo que: \\

\begin{itemize}

\item $\alpha = 0.25$

$\mu_x \in$ [371.62187 , 390.86792]

\item $\alpha = 0.15$

$\mu_x \in$ [369.20278 , 393.28702]

\item $\alpha = 0.05$

$\mu_x \in$ [364.84920 , 397.64060]

\end{itemize}


La diferencia es a nivel decimal, por lo tanto no afecta tanto, en situaciones aproximadas.

\item Calcular el intervalo de confianza de tiempo promedio de los cigarros con filtro color caf\'e, con un 75 \%,
85 \% y 95 \% de confianza.

Teniendo en cuenta que: \\

$P(\overline{y} - t_{1-\frac{\alpha}{2}}\frac{\widehat{\sigma}}{\sqrt{n}} <= \mu_y <= \overline{y} + t_{\frac{\alpha}{2}}\frac{\widehat{\sigma}}{\sqrt{n}}) = 1 - \alpha$\\
$\overline{y} = 399.6327$\\
$\widehat{\sigma} = 48.83462$\\
$n = 50$\\
t = Distribucion t-student con n-1 grados de libertad\\

Por lo que: \\

\begin{itemize}

\item $\alpha = 0.25$

$\mu_y \in$ [391.50907 , 407.75624]

\item $\alpha = 0.15$

$\mu_y \in$ [389.42660 , 409.83871]

\item $\alpha = 0.05$

$\mu_y \in$ [385.60571 , 413.65960]

\end{itemize}

\item %
% PARECE QUE ESTA LISTA, REVISAR
%

La edad a la que un hombre contrae matrimonio por primera vez es una variable aleatoria con distribuci\'on gamma.
Si la edad promedio es de 30 a\~nos y los mas com\'un es que el hombre se case a los 22 a\~nos.
% Gamma
% Media = 30 anios
% Moda = 22 anios 
\begin{itemize}
	\item Encontrar los par\'ametros de forma y escala de la distribuci\'on.\\

	Aca nos interesa encontrar cuanto es el tiempo esperado en que un hombre se case.
	Nos dicen que la edad promedio (el cual podemos considerar como nuestro $\lambda$) es de 30 a\~nos,
	y como nos interesa saber cuando un hombre se casa ''por primera vez``, tomamos a $\alpha = 1$. Ahora:
	Sabemos que la media de una distribuci\'on $\Gamma$ es igual a:
	$$\bar{x}\ =\ E(x)\ =\ \frac{\lambda}{\alpha}\ =\ 30$$
	y sabemos que la moda de una distribuci\'on $\Gamma$ es igual a:
	$$Moda\ =\ \frac{\lambda-1}{\alpha}\ =\ 22\ \ \ \ \ si\ \lambda>1$$
	Por lo que resolviendo el sistema de ecuaciones, nos queda $\lambda = 26.5$\\
	
	$$\therefore\ p(t)\ =\ \frac{\lambda^{\alpha}t^{\alpha-1}e^{-\lambda t}}{\Gamma(\alpha)}\ =\ \frac{26.5\cdot e^{-t}}{\Gamma(1)} $$
	
	\item ?`Cu\'al es la probabilidad de que un hombre se case despu\'es de los 25 a\~nos?\\
	$$\therefore\ p(t>25)\ =\ 1-p(t<25)\ =\ 1\ -0.6106966\ =\ 0.3893034$$

	\item Grafique la funci\'on de densidad de probabilidad y de distribuci\'on.\\\\
	Funci\'on de densidad de probabilidad:\\
  	  \includegraphics[width=3.3in,height=3.3in]{images/2_5-dgamma.png}\\
	Funci\'on de distribucion:\\
  	  \includegraphics[width=3.3in,height=3.3in]{images/2_5-pgamma.png}
	
\end{itemize}

\item Calcular el intervalo de confianza para la varianza del tiempo de duraci\'on del cigarro de tipo light, con
un 75 \%, 85 \% y 95 \% de confianza.

Teniendo en cuenta que: \\

$P(\frac{\sum(x_i - \overline{x})^2}{\chi_{1-{\frac{\alpha}{2}}}^2} <= \sigma^2 <= \frac{\sum(x_i - \overline{x})^2}{\chi_{\frac{\alpha}{2}}^2}) = 1 - \alpha$\\

$\widehat{s}^2 = \sum \frac{(x_i - \overline{x})^2}{n-1}$\\

$\Leftrightarrow P(\frac{\widehat{s}^2 (n-1)}{\chi_{1-{\frac{\alpha}{2}}}^2} <= \sigma^2 <= \frac{\widehat{s}^2 (n-1)}{\chi_{\frac{\alpha}{2}}^2}) = 1 - \alpha$

$\widehat{s}^2 = 3428.939$\\
$n - 1 = 49$\\
$\chi^2$ = distribucion chi-cuadrado con n-1 grados de libertad\\

Por lo que: \\

\begin{itemize}

\item $\alpha = 0.25$

$\sigma^2 \in$ [2770.269 , 4446.106]

\item $\alpha = 0.15$

$\sigma^2 \in$ [2623.445 , 4744.634]

\item $\alpha = 0.05$

$\sigma^2 \in$ [2384.568 , 5351.706]

\end{itemize}

\item Calcular el intervalo de confianza para la varianza del tiempo de duraci\'on del cigarro con filtro color
caf\'e, con un 75 \%, 85 \% y 95 \% de confianza.

Teniendo en cuenta que: \\

$P(\frac{\sum(y_i - \overline{y})^2}{\chi_{1-{\frac{\alpha}{2}}}^2} <= \sigma^2 <= \frac{\sum(y_i - \overline{y})^2}{\chi_{\frac{\alpha}{2}}^2}) = 1 - \alpha$\\

$\widehat{s}^2 = \sum \frac{(y_i - \overline{y})^2}{n-1}$\\

$\Leftrightarrow P(\frac{\widehat{s}^2 (n-1)}{\chi_{1-{\frac{\alpha}{2}}}^2} <= \sigma^2 <= \frac{\widehat{s}^2 (n-1)}{\chi_{\frac{\alpha}{2}}^2}) = 1 - \alpha$

$\widehat{s}^2 = 2384.821$\\
$n - 1 = 49$\\
$\chi^2$ = distribucion chi-cuadrado con n-1 grados de libertad\\

Por lo que: \\

\begin{itemize}

\item $\alpha = 0.25$

$\sigma^2 \in$ [1926.717 , 3092.259]

\item $\alpha = 0.15$

$\sigma^2 \in$ [1824.601 , 3299.884]

\item $\alpha = 0.05$

$\sigma^2 \in$ [1658.463 , 3722.101]

\end{itemize}

\item ?`Se puede decir que el cigarro light presentan mayor duraci\'on?. fundamente, use $\alpha$ = 0,05.
	\begin{itemize}
		\item Como podemos apreciar, comparando los intervalos de confianza de tiempo promedio de los cigarros light y corrientes. Los cigarros corrientes presentan un limite superior e inferior mayores que los cigarros light, con una diferencia de unos 20 segundos app. Esto quiz\'as se deba a que al ser mas fuerte el cigarro corriente, la gente se demora mas en terminarlo por lo general.\\\\
	Intervalos de Confianza de tiempo Promedio\\
        \begin{tabular}{|c|c|c|}\hline

         & Cigarro Light & Cigarro Corriente\\\hline
        $\mu$ & [$364$ , $398$] & [$391$ , $407$] \\\hline
        \end{tabular}

	
	\end{itemize}

\item Escriba conclusiones a partir de los resultados anteriores, tomen en cuenta el largo del intervalo y si
este contiene al cero en el caso de diferencia de medias.

Analizamos los intervalos de confianza al 95\% con $n=50$ :

        \begin{tabular}{|c|c|c|}\hline
		
	 & Cigarro Light & Cigarro Corriente\\\hline
	$\overline{x}$ & $381.2$ & $399.6$ \\\hline
	$\widehat{\sigma}$ & $58.5$ & $48.8$  \\\hline
	$\mu$ & [$364$ , $398$] & [$391$ , $407$] \\\hline
	$\bigtriangleup\mu$ & $33.6$ & $28$ \\\hline
	$\sigma^2$& [$2348$ , $5351$]  & [$1658$ , $3722$]  \\\hline
	$\bigtriangleup\sigma^2$& $3003$ & $2064$ \\\hline
	$\sigma$& [$48$ , $73$]  & [$40$ , $61$]  \\\hline
	$\bigtriangleup\sigma$& $25$  & $21$ \\\hline
	\end{tabular}

y vemos la diferencia de medias con 95\% de confianza:

$\mu_x - \mu_y = (\overline{x} - \overline{y}) \pm t_{\frac{\alpha}{2}} \cdot \widehat{\sigma}^2$\\
con : $\widehat{\sigma}^2 = \sqrt{\frac{1}{n_A}+\frac{1}{n_B}}\cdot\sqrt{\frac{n_A \widehat{\sigma}_{A}^{2} + n_B \widehat{\sigma}_{A}^{2}  }{n_A + n_B - 2}}$\\
$\mu_x - \mu_y = -18 \pm 1.985 \cdot 10.77$\\
$\mu_x - \mu_y = -18 \pm 21.3859$\\
$\mu_x - \mu_y =$ [$-39.38$ , $3.38$]\\

\begin{itemize}
	\item Se puede ver en la diferencia de medias, que esta la probabilidad de que las medias sean iguales. Esto probablemente cambie si es que consideramos una mayor cantidad de datos medidos (mayor a 150), lo cual nos daria un rango mas cercano al comportamiento real de la distribuci\'on.
	\item Podemos concluir que se ve una peque\~na notable diferencia entre los tiempos en que se demora una persona en fumarse un cigarro light y uno corriente. 
	A pesar de haber obtenido una relaci\'on entre el tiempo de fumado y el tipo de cigarro, los rangos de los intervalos son muy similares, por lo que se dislumbra que ambas distribuciones comparten el mismo comportamiento, pero con la media corrida en aproximadamente 20 seg, por lo que podemos concluir que los cigarros light tienen menor duracion, en general, con los cigarros corrientes.
\end{itemize}


\end{enumerate}

\end{document}
