\begin{center}
	\includegraphics[height=1.5cm]{images/umbrello}\\
\end{center}
\textbf{Umbrello} es una herramienta libre para crear y editar diagramas UML, que ayuda en el proceso del desarrollo de software. Fue desarrollada por Paul Hensgen, y está diseñado principalmente para KDE, aunque funciona en otros entornos de escritorio.\\

Umbrello maneja gran parte de los diagramas estándar UML pudiendo crearlos, además de manualmente, importándolos a partir de código en C++, Java, Python, IDL, Pascal/Delphi, Ada, o también Perl (haciendo uso de una aplicación externa). Así mismo, permite crear un diagrama y generar el código automáticamente en los lenguajes antes citados, entre otros. El formato de fichero que utiliza está basado en XMI.\\

También permite la distribución de los modelos exportándolos en los formatos DocBook y XHTML, lo que facilita los proyectos colaborativos donde los desarrolladores no tienen acceso directo a Umbrello o donde los modelos van a ser publicados vía web.Umbrello se distribuye en el módulo kdesdk de KDE.\\

\textbf{Web:} http://uml.sourceforge.net/

