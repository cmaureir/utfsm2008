\begin{itemize}
	\item Casi todas las condiciones para desarrollar algun proceso, fueron plasmadas con un ``include'' en los casos de uso para que se verificara la
		condicion antes de realizar el acto.
	\item Se establecieron los diferentes casos de uso en sectores diferentes del diagrama para notar el conjunto de funcionalidades de cada
		requerimiento
	\item En el modelo estatico, se utilizaron flechas direccionales para facilitar la lectura visual de las relaciones, y se agruparon las funcionalidades como ingrsar, modificar, inscribir, etc. para poder simplificar el diagrama. Tambien, se crearon clases que pueden sonar que repiten informacion (como obsequio y regalo), pero que se identifican con procesos diferentes, relacionandose con diferentes entidades y por ende, fueron identificados como entidades aparte.
\end{itemize}

