\begin{itemize}
	\item P\'agina 3: P\'ongase usted en el lugar de Del Vecchio. \\
	\begin{enumerate}
		\item ?`C\'omo podr\'ia usar un an\'alisis FODA -fortalezas, debilidades, oportunidades y amenazas- para administrar estrat\'egicamente este imperio mundial en perfecta expansi\'on?\\
		Bueno, el an\'alisis FODA al analisar las amenazas y debilidades de la compa\~nia junto con las fortalezas y oportunidades, permite vislumbrar el real marco de la empresa en el mercado. De esta forma, la empresa puede planear estrategias adecuadas a la realidad de la compa\~nia y proyectarse mejor a futuro. Especialmente, al crecer y expandirse en diferentes tipos de mercado, nececitaria realizar un FODA para cada area, y asi enfocar sus esfuerzos en las que tengas mas oportunidades y reforzar las mas debiles si es que a futuro se ven prometedoras. Aun asi, el an\'alisis FODA no es la panacea para todos los problemas, pero si es una herramienta que ayuda a la toma de decisiones y guiarlas por un buen camino. 

		\item ?`Qu\'e revelar\'ia un an\'alisis FODA acerca de la situaci\'on de Luxottica?.\\
		Revelar\'ia las diferentes oportunidades de expansion para poder desarrollandose y expandiendose como compa\~nia, las debilidades de cada una de sus areas, las fortalezas que la mantienen como uno de los principales vendedores de productos opticos en el mundo y contrastandolo con las posibles amenazas tiene actualmente en el mercado.	Asi podr\'a darse cuenta de que deber\'a priorisar, si conquistar otra empresa relacionada con la optica para reducir la amenazante competencia, o mejorar la calidad o cantidad de producci\'on dentro de una de sus areas, o simplemente manternerse como esta gracias al gran bienestar econ\'omico de la compa\~nia. La eleccion esta en el gerente.
	\end{enumerate}

	\item P\'agina 15\\\\
	Un gerente asistente que realiza su trabajo en una tienda que pertenece a una cadena de almacenes de descuento que se expande
	con rapidez, est\'a a punto de inagurar un nuevo local. Obviamente la inauguraci\'on de la nueva tienda implica mucho trabajo, como
	pedir mercanc\'ia, inspeccionarla a su llegada, colocarla en los anaqueles, y as\'i sucesivamente. El Gerente cuenta con cinco
	ayudantes que fueron transferidos desde otras tiendas de la cadena. A esas personas se les dijo que se trataba de un programa de
	"capacitaci\'on administrativa". Las seis personas trabajaron muy arduamente y la tienda se inaugur\'o en la fecha prevista. Una
	semana despu\'es de la inauguraci\'on, su superior directo del Gerente le ordena que encuentre una buena raz\'on para deshacerse de
	tres de esos cinco empleados, porque solamente hay lugar para dos personas.\\
	Al respecto caben, sin dudas las siguientes interrogantes:\\
	\begin{enumerate}
		\item ?`Considera usted que sea malo dejar cesantes a esas personas, despu\'es de haberles dicho que participar\'ian de un programa de
capacitaci\'on administrativa?.\\
	Por supuesto que si, utilizar a las personas como si fueran simples herramientas para lograr algo, es cruel e inhumano. Aun asi, es un caso muy com\'un y ``aceptado'' comunmente por la sociedad, especialmente las personas que nos les queda mas que trabajar sin un contrato.

		\item ?`Qu\'e dilemas \'eticos observa en este caso?.\\
	Aca nos encontramos con el dilema de de discriminar sin razon a una persona para dejarla fuera de un trabajo, y utilizar falsas verdades o mentir descaradamente para justificar tales acciones. No es algo facil de lidiar, y el peso de jugarse el destino laboral de las personas no es algo que todas las personas pueden soportar. Al final, las personas terminan insensibilizandose para cumplir los mandatos de un jefe y asi mantener su puesto de trabajo, en vez de denunciar estos tratos injustos y terminar metido en un gran problema.

		\item ?`C\'omo resolver\'ia usted esta situaci\'on?.\\
	Estas situaciones se pueden evitar. Basta con ser correcto, y plantear las cosas como son desde un comienzo. Al momento de contratarlos, les hubiera planteado que al finalizar las obras, solo podran continuar en el proyecto numero limitado de personas, y que dependiendo de su desempe\~no, seran elegidos finalmente los mas aptos. Otra opcion, aunque no la encuentro muy justa tampoco, seria dejar 2 personas mas aptas contratadas, y a las otras 3 ofrecerles la realizacion de otro proyecto o trabajo gracias a su buen desempe\~no, y que desean mantenerlos mas activos en la compa\~nia. Encontrar una opcion justa para todos es muy dificil si es que uno se pone a buscar una solucion cuando ya todo esta realizado. Por ello, es muy importante aprender a preveer estas situaciones, y llevar a cabo un plan acorde al futuro.
	\end{enumerate}

	\item P\'agina 22: Reflexiones y discuci\'on sobre cuestiones relevantes de administrac\'ion estrategica.
	\begin{enumerate}
		\item Realice un an\'alisis FODA de un negocio local que usted crea conocer bien. ?`Qu\'e ventaja competitiva, si la hay, distingue a esta organizaci\'on de las dem\'as?\\
	Un ejemplo tomando en cuenta un hospital privado como la cl\'inica ren\~naca (algunas cosas las supuse para poder realizar un an\'alsis mas interesante):\\
\begin{enumerate}
		\item Fortalezas
\begin{itemize}
	\item Planta medica de ato nivel.
	\item Centro de referencia regional, cuenta con todas las especialidades. S.S. Concepci\'on cubre Santa Juana, Florida, Lota, Coronel, Hualqui y el Hospital Guillermo Grant Benavente.
	\item Compromiso de los m\'edicos con su servicio.
	\item Convenio Docente Asistencial.
	\item Atenci\'on exclusiva de transplantes.
	\item Continua Capacitaci\'on mediante jornadas y seminarios.
	\item Servicios de apoyo cl\'inico y terap\'eutico.
	\item Numero de prestaciones anuales:
	\begin{enumerate}
		\item Atenci\'on Abierta: policlinicos
		\item Atenci\'on Cerrada: hospitalizaci\'on.
		\item Unidades de Cuidado Intensivo.
	\end{enumerate}
\end{itemize}
		\item Debilidades:
\begin{itemize}
	\item Forma de contrato con la gente: tienen contrato indefinido, es decir, no existe evaluaci\'on de resultados, mas bien se controla presencia y puntualidad, pero no la efectividad ni calidad de servicio.
	\item M\'edicos liberados de guardia: si un medico ha prestado servicios de urgencia por mas de 20 a\'unos, se libera de la obligaci\'on contractual de hacer horas de urgencia, y se le sigue remunerando por ello.
	\item Falta de comunicaci\'on con unidades perif\'ericas de comunas distantes.
	\item Falta de agilidad, tramite burocr\'atico.
	\item Atenci\'on muy costosa para el paciente.
\end{itemize}
		\item Oportunidades:
\begin{itemize}
	\item Aplicaci\'on de avances m\'edicos desarrollados por las distintas Universidades y los especialistas que trabajan en el hospital.
	\item Desarrollar el \'area de Pensionados para aumentar ingresos propios.
	\item Problemas y desconfianza del sistema publico de salud ocasionado por diferentes huelgas por las malas situaciones de trabajo.
\end{itemize}
		\item Amenazas:
\begin{itemize}
	\item Crisis econ\'omica que afecta de dos formas:
	\begin{enumerate}
	\item Aumento explosivo de pacientes (traslado Isapres- Fonasa)
	\item Disminuci\'on del presupuesto del Hospital.
	\end{enumerate}
	\item Nuevos virus y enfermedades. 
\end{itemize}
\end{enumerate}

		\item ?`Qu\'e diferencias habr\'ia en los procesos de formulaci\'on e implementaci\'on de una estrategia en los siguientes casos:\\
		 (a)en negocios grande, (b) en negocios peque\~nos, (c) en organizaciones sin fines de lucro y (d) en negocios mundiales?\\
		
		Bueno, cada tipo de negocio debe manejar y tener en cuenta un punto de vista diferente de ver las oportunidades y amenazas del ambiente. Esto se debe a causa de que cada parte maneja un segmento diferente del mercado, y por ende ciertas cosas le afectan mas a unos que a otras. Tipificando cada uno:\\
	\begin{enumerate}
		\item Negocios grandes:\\
			Su negocio reside en ser proveedores de los negocios peque\~nos, por lo que producen masivamente y tienen que saber innovar y llevar la cabeza en el mercado. Necesitan estar siempre un paso mas all\'a del resto, y tener mucho cuidado en sobre vender algun producto y quedarse con demasiados excesos de la producci\'on. Tienen que tener especial atenci\'on en la capacidad de las peque\~nas empresas de pedir prestamos y mantener una solvencia en el mercado. Para poder diferenciarse del resto de los proveedores, sera muy importante establecer una estrategia competitiva clara en frente al resto, sea en costos, diferenciandose por sus productos de calidad, etc. Constantemente deben desarrollar cosas nuevas y mantenerse analisando el entorno competitivo.
		\item Negocios peque\~nos:\\
			Estos, al poseer una gran cantidad de competidores, a traves de estrategias corporativas de estabilidad deben posicionarse en el mercado para luego, al disminuir sus competidores o tener una ventaja que le permita destacar, aprovechar las ventajas competitivas para crecer como compa\~nia. Por ello, deben enfocarse en el an\'alisis de su entorno para poder salir adelante, y mantenerse constantemente vigilando a los competidores para poder anticiparse a los cambios de las tendencias del mercado.
 
		\item Organizaciones sin fines de lucro:\\ Se deben enfocar principalmente en mantener una misi\'on, objetivos y estrategias claras para toda la organizaci\'on, manteniendo una moral alta en todos sus miembros. Mas que enfocarse en la competencia de otras empresas, tiene que poder maneterse estable y lograr el apoyo de la comunidad para su sustentabilidad.
		\item Negocios mundiales:\\ Estos se encuentran en un ambiente din\'amico, donde existe mucha incertidumbre de cuales seran las nuevas tendencias y tecnolog\'ias que iran se la parte vital en la dominaci\'on del mercado. Por ello, la administraci\'on estrat\'egica es lo mas importante que deben llevar a cabo, para proveer a los gerentes de un medio sistem\'atico y completo para analizar el entorno, evaluar fortalezas y debilidades de su propia organizaci\'on, etc. Al ser organizaciones de gran complejidad, es necesariomantener estrategias corporativas de combinaci\'on dependiendo de cada sector o negocio manejado. 
	\end{enumerate}

		\item Dado que existen diferencias en el tipo de decisiones que toman los gerentes en los niveles alto, medio y bajo. ?`C\'omo cree usted que se aplica esta jerarqu\'ia a los tres niveles de la planificaci\'on estrat\'egica?\\
		Al principio uno tender\'ia a establecer una relaci\'on 1 es a 1, entre los niveles de la planificaci\'on estrategica y claramente
		as\'i es la soluci\'on. Si pensamos en un Gerente de un nivel alto, todas las desiciones que el considere no tienen que ser tan 
		pr\'acticas, al contrario deben ser de la forma en la cual se intente determinar qu\'e negocios debe envolver la organizaci\'on
		entonces lo principal que el debe preocuparse es tratar de decidir \emph{los negocios en los cuales se tiene que incursionar}, por
		lo tanto sus deciciones son clave y marcan el norte de la empresa.
		Por otro lado, tomamos en cuenta a un Gerente de nivel medio, podemos asociarlo de inmediato al nivel de negocios, ya que lo que har\'a
		es intentar determinar \emph{como competir la organizaci\'on en cada sector donde se tiene el negocio}, por lo tanto su funci\'on es poder
		llevar a cabo y mantener la desici\'on tomada por el gerente de nivel alto.
		Por \'ultimo falta la tarea de poder \emph{determinar la forma de respaldar la estrat\'egia a nivel de negocio}, trabajo que le
		corresponde a un Gerente de nicel bajo, ya que esto es como el sistema circulatorio de la organizaci\'on, debe preocuparse de
		el marketing, la producci\'on, recursos humanos, investigaci\'on, etc.

		\item Arie P. DeGues, quien dirige las actividades de planificaci\'on de las compa\~n\'ias integrantes del Royal Dutch/Shell Group, declar\'o en una ocasi\'on que la capacidad de ``aprender m\'as r\'apidamente que los competidores'' podr\'ia ser la \'unica ventaja competitiva sustentable.\\
		 ?`Est\'a usted de acuerdo con esa idea? ?`Por qu\'e si o por qu\'e no?\\\\
	La mayoria de las veces, si. ?`Y por qu\'e pensamos eso?\\ Porque a excepci\'on de algunas pocas compa\~nias, el poder aprender y aprovechar mas rapidamente que la competencia las distintas innovaciones y tecnologias, es una de las mas grandes y sustentables en el tiempo. Esta ventaja es muy importante al formar la visi\'on que las personas tienen de la empresa, ya que las personas tienden a relacionar una empresa bien capacitada y confiable con la primera que invente e implemente las \'ultimas tecnolog\'ias, diferenciandolas notablemente de la competencia. No importa que tan buena y estable sea el pedazo de mercado que posea la compa\~nia, si es que la competencia logra desarrollar y implementar antes y mejor una nueva invenci\'on que llame la atenci\'on de la comunidad, bien manejada, puede llegar quitarles gran parte de la clientela a las que no lo hicier\'on a tiempo. Esta ventaja competitiva, esta muy relacionada con la capacidad de innovar y crear nuevas soluciones para la sociedad.\\
	Si uno anal\'iza la realidad actual, la mayor\'ia de las compa\~nias ``top'' del mercado, son las que continuamente van desarrollando nuevos productos y mejoras a sus servicios. Inclusive, hay muchas empresas jovenes que han emergido gracias a esa \'unica caracter\'istica, entrando fuertemente al mercado globlal.\\
	A pesar de todo, el mantener esta ventaja, es una de las metas mas dif\'iciles de conseguir, ya que muchas veces signif\'ica realizar inversiones de alta peligrosidad, intentando proyectar las necesidades futuras de las personas. No todas las empresas logran este cometido, y se vuelven dependientes de otras compa\~nias mas grandes o astutas.

		\item Cite cinco ejemplos de lo que usted considera que es una UEN.Para cada uno de sus ejemplos, explique por qu\'e cree que cada uno es una UEN.\\
		Entendemos UEN como Unidad Estrat\'egica de Negocios.En cada punto estara la organizaci\'on y en el p\'arrafo se explicara la UEN:		\begin{enumerate}
			\item[LeSancy]:\'Esta organizaci\'on es una tradicci\'on con lo que respecta a higiene, por como impusieron el concepto
				de jab\'on, con su forma particular. Hace poco lesancy lanz\'o al mercado una variedad de Shampoo y Acondicionadores
				que rompe con la tradici\'on de lo que era la marca, es por eso que ahora posee dos UEN, una que est\'a enfocada
				a lo que es el negocio con respecto a los jabones y otra UEN con relaci\'on a los Shampoo y acondicionarores.Entonces
				si pensamos un poco es obvio que deben una UEN para cada caso, porque apuntan a estrat\'egias distintas y la forma
				del negocio que se desarrolla a su entorno, son distintas.
			\item[Coca-Cola]:Marca mundialmente conocida que posee muchos productos denstro de su organizaci\'on, tenemos el caso de los
				refrescos (Coca-Cola, Fanta, Sprite, ...), pero por otro lado tenemos lo que es KAPO, Aguas Minerales, Bebidas
				energ\'eticas las cuales claramente deben poseer una UEN cada uno, porque el \'ambito del negocio es distinto, 
				la estrat\'egia con la que se desarrolla cada uno de estos productos es distinta. Hasta el publico al cual est\'a
				enfocado cambia la estrat\'egia de negocio.
			\item[BIC]:La organizaci\'on que est\'a detr\'as de \'esta marca, posee una ardua tradici\'on en lo que son los l\'apices
				pero como todos sabemos tambien existen afeitadoras de esta misma marca, por lo tanto el negocio es totalmente distinto
				y nuevamente es necesario tener Una UEN por cada tipo de negocio, porque apuntan a distintas cosas y la estrat\'egia deben
				ser muy distintas y enfocadas al negocio que se desea sostener
		\end{enumerate} 
		En general todas las UEN nombradas anteriormente cumplen los mismos principios que se se\~nalan en \'este texto, son tipos de carteras
		de negocios, teniendo bien definidos los segmentos del producto-mercado, con su estrat\'egia claramente definida.
		Cada unidad, posee una estrategia dependiendo de las capacidades y necesidades del negocio.
		Y por ultimo cada cartera es administrada para el fin de los intereses de la organizaci\'on en conjunto
	\end{enumerate}

	\item P\'agina 23
	\begin{enumerate}
		\item ?`Qu\'e ventaja(s) competitiva(s) considera usted que posee Gibson Guitars?. Explique su respuesta.\\
		Los musicos aun cuando la empresa se encontraba en crisis, consideraban
 la calidad de las guitarras GIbson, entre las mejores disponibles, lo que implicaba un
 prestigio intacto frente a sus competidores.\\

		La empresa pese a todas las mejores tecnologicas que fueron necesarias
 para su resurgimiento, posee ese toque artesano que lo caracteriza frente al resto.\\

		Los empleados que realizan las pruebas de control de calidad a las guitarras
 son musicos, lo que garantiza mas que solo un correcto funcionamiento tecnico.\\

		Por ultimo, la empresa esta completamente abierta a innovaciones que mejoren 
el prestigio, ejemplo de esto es la venta de guitarras clasicas.\\

		\item ?`Cu\'al es la estrategia competitiva de Bibson Guitars?. Explique su respuesta.\\
		La empresa Gibson presenta una estrategia competitiva del tipo "Diferenciaci\'on", ya 
que aplica ingenieria a los productos, utilizando tanto maquinaria altamente tecnol\'ogica, como artesanal,
 ademas de sistemas computacional que dan soporte a ello. Presenta un prestigio corporativo adquirido 
mediante muchos a~nos, como tambien larga tradicion reconocida por los m\'usicos. Posee mano de obra
 creativa, los que realizan los controles de calidad y altamente calificados debido a su profesi\'on
 en el mismo rubro. Tambi\'en posee habilidades de marketing aprovechando su gran prestigio durante
 a~nos, logrando atraer a nuevos clientes, con productos famosos en el tiempo, como lo son las 
guitarras cl\'asicas.\\

		\item ?`Qu\'e rol podri\'ia desempe\~nar la GCT en el desarrollo de una ventaja competitiva para Gibson Guitars?.\\
		La  Gesti\'on de la calidad en Gibson, podria ser aplicada logrando calidad y mejoramiento
 continuo tanto en la fabricaci\'on de nuevos tipos de guitarra, con ese toque artesanal que lo caracteriza, 
junto al despliegue tecnico necesario para la realizaci\'on en si de los productos. El objetivo final ser\'ia
 satisfacer la necesidad de calidad de los m\'usicos que prefieren a Gibson, logrando as\'i diferenciarse de
 sus competidores, atraer a nuevos clientes en busca de buenos productos y continuar calidad, fiabilidad
 y prestigio, tanto continuo como en alza.\\

		\item ?`C\'omo pudo haber aplicado Juszkiewicz el an\'alisis FODA cuando lleg\'o por primera vez a Gibson Guitars?. ?`Que tipo de elementoshabr\'ia buscado al realizar un an\'alisi FODA?.\\
		Juszkiewicz debi\'o haber llegado a Gibson, con un analisis FODA previamente realizado
 o al menos con alguno de los puntos, debido a que si bien se sab\'ia que a la empresa no le iba bien,
 si bien una cuota de riesgo es bien visto, no se deben tomar sin tener un estudio minimo que lo
 soporte. Al comprar Gibson, claramente debio notar las fortalezas que la empresa aun manten\'ia
 durante los a~nos de prestigo, como tambien las debilidades que tenia al presentar metodos de 
producci\'on poco eficientes, lo que lo llevo a modernizar toda la empresa. Las oportunidades estaban
 presentes aun, de la mano de las fortalezas y metodos de producci\'on mas eficientes, solo habia que
 analizar el nicho en que se encontraba. Todo esto implica obviamente tener en cuenta las amenazas 
que les provee el medio, para asi generar buenas estrategias.\\

		\item ?`Describa los tipos de -grandes- estrategias corporativas que ha observado en este caso.\\
		Primero nos podemos dar cuenta de que la primera estrategia fue hacer un cambio de ``switch'' a lo que era la empresa
		los nuevos due\~nos se dieron cuenta que si no hac\'ian algo r\'apido la compa\~nia iba a sucumbir, es por eso que
		la primera tarea fue reducir los costos, se se\~nala en el texto que el sistema de procesamiento de datos de \$500000
		fue sustitu\'ido por un equipamiento computacional de \$15000, con lo cu\'al el ahorro es notorio. Por otro lado se comenz\'o
		a despedir gente, sustituyendola con maquinaria que hiciese el mismo trabajo, todo con la idea de \emph{eficiencia y modernizaci\'on}
		Lo importante es que la estrat\'egia que mantuvo esta empresa en pie, fue la de no dejar atr\'as su tradici\'on de la misma
		fabricaci\'on de las guitarras, ya que eso le daba el ``toque'' a la empresa.
		
	\end{enumerate}
	

	\item P\'agina 24:
	\begin{enumerate}
		\item ?`Charlie desea que el crecimiento de Megatoys continu\'e. Como podr\'ia aplicar los conceptos de la administraci\'on estrat\'egica como una ayuda para lograr sus metas?\\\\

		Charlie debe identificar una nueva misi\'on, objetivos y estrategias para su nueva Empresa, debido a que si se los hermanos se separaron de ABC Toys, debe haberse producido por que quer\'ian cambiar de aire y haber identificado un nuevo nicho en donde poder concentrarse, obviamente esto requiere una nueva estructuraci\'on a lo que estaba presente en ABC Toys.\\
		Tuvieron que haber analizado el ambiente actual del rubro y haberse proyectado suficientemente bien para haber tomado la desici\'on de separarse, lo que implica haber identificado nuevas oportunidades, diferentes a las que cubr\'ian su anterior marca, para evitar amenazas que involucra el sector, las cuales deben conocer por los a~nos de experiencia en \'el.\\
		En cuanto a recursos se refiere, debieron haber conseguido los suficientes con todo esos a~nos de \'exito, para poder tener una base con la cual comenzar su nueva idea.\\
		Identificar las fortalezas de su nueva empresa debe ser primordial, como puede ser la experiencia en el rubro, como su prestigio. Por el contrario identificar las debilidades que su nueva propuesta puede traerles, como la incursi\'on en una \'area de innovaci\'on totalmente desconocida, lo que puede ser un arma de doble filo.\\
		Finalmente debe formular estrategias para los niveles jer\'arquicos de la empresa, para que sean los encargados de seleccionar las estrategias que sean compatibles con cada nivel y poder aprovechar de forma solvente las fortalezas y oportunidadas descubiertas con anterioridad.\\

		\item?`Seria \'util para Charlie el an\'alisis FODA en la administraci\'on de Megatoys? ?`Por que si o por que no?. Explique su respuesta.\\\\
		Claramente debe rehacer un analisis FODA en Mega Toys, debido a que si su motivo de separaci\'on de ABC Toys fue por poseer nuevas ideas, debe reestructurar todo nuevamente, desde su nueva posible misi\'on, pasando por descubrir y describir su fortaleza organizacional, oportunidades en el nuevo nicho identificado, implicando tambien asumir o mejorar sus debilidades y tratar de tenerlas en mente a la hora de formular estrategias y por \'ultimo tener una capacidad de vision bastante amplia, para descubrir las amenazas que esta nueva incursi\'on les pueda implicar, solo as\'i podran establecer nuevas estrategias para Mega Toys.\\

		\item Charlie le ha solicitado a usted que haga una presentaci\'on para sus empleados sobre el tema de la ventaja competitiva. Elabore una lista de las principales ideas que desearia usted comunicarles.\\\\

		- La ventaja competitiva es un concepto clave en la administraci\'on estrat\'egica.\\
		- Distingue a una organizacion del resto, logrando sacar a luz las capacidades, ya sea organizacionales, las cuales no estan presentes o son inferiores en otras organizaciones o referente a los activos que posee en menor cantidad o no posee la competencia.\\
		- Se logra aprovechando de manera eficaz los sistemas y recursos que posee la organizaci\'on.\\
		- No es suficiente alcanzar esta ventaja competitiva, sino que el trasfondo de la idea es sostenerla y mejorarla.\\
		- El sustento de esta ventaja se logra mejorando la calidad, fiabilidad e inovaci\'on de los productos, junto al satisfacer la necesidad de calidad que poseen los clientes, logrando as\'i nuevos clientes leales.\\

	\end{enumerate}
\end{itemize}
