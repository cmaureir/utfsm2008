\begin{enumerate}
	\item ?`Qu\'e se etiende por FODA?

        Primero tenemos que saber que sus siglas son, Fortalezas, Debilidades, Oportunidades y Amenazas.

	El fin principal de \'este an\'alisis es el localizar un nicho estrat\'egico que la organizaci\'on sea capaz de explotar.

	En otras palabras el objetivo final del an\'alisis FODA es poder determinar las ventajas competitivas que tiene la empresa
	bajo an\'alisis y la estrategia gen\'erica a emplear por la misma que m\'as le convenga en funci\'on de sus caracter\'isticas propias
	y de las del mercado en que se mueve.

	Gracias al an\'alisis, los gerentes revaloran tambi\'en la misi\'on y los objetivos actuales de su organizaci\'on, generalmente en
	torno a las siguientes interrogantes: ¿son realistas?; ¿requieren alguna modificaci\'on?; ¿son tal como se desean en este momento?.
	Si se requieren cambios en la direcci\'on general, al final de este an\'alisis es donde probablemente se originar\'an. Si no se necesita
	cambio alguno, entonces la gerencia ya est\'a preparada para iniciar la formulaci\'on real de sus estrategias.

	\item ?`Cu\'ales son los distintos niveles de estrategia?

	\begin{itemize}
		\item \emph{Estrategia a nivel corporativo}

		Es aquella que intenta determinar en qu\'e negocios debe envolverse una organizaci\'on.
		
		Si una organizaci\'on realiza m\'as de un tipo de negocios, necesitar\'a una estrategia a nivel
		corporativo. Dicha estrategia intenta responder ala siguiente pregunta: ¿en qu\'e negocio o
		negocios debemos incursionar? La estrategia a nivel corporativo determina los roles que cada
		unidad de negocios de la organizaci\'on habr\'a de desempe\~nar. En una compa\~n\'ia como PepsiCo, por ejemplo,
		la estrategia a nivel corporativo de la gerencia integra las estrategias de sus divisiones Pepsi,
		7-Up International y Frito-Lay. PepsiCo ten\'ia una divisi\'on de restaurantes que inclu\'ia Taco Bell,
		Pizza Hut y KFC, pero, por las intensas presiones competitivas que existen en la industria de
		restaurantes, PepsiCo cambi\'o su estrategia a nivel corporativo y vendi\'o esa divisi\'on para concentrarse
		en sus divisiones de bebidas y bocadillos.
		\item \emph{Estrategia a nivel de negocios}

		Es aquella que intenta determinar como debe competir una organizaci\'on en cada uno de sus negocios.
	
		 Una estrategia a nivel de negocios intenta responder la siguiente pregunta: ¿c\'omo se debe competir en cada uno de los negocios
de la organizaci\'on?. Cuando se trata de una organizaci\'on peque\~na que s\'olo tiene una l\'inea de negocios, o de una organizaci\'on
grande que no se ha diversificado en diferentes productos o mercados, la estrategia a nivel de negocios coincide con la estrategia de
la organizaci\'on a nivel corporativo. En cambio, si se trata de una organizaci\'on con m\'ultiples negocios, cada divisi\'on tendr\'a su propia
estrategia con la cual definir\'a los productos o servicios que ofrece, los clientes en quienes desea incidir, y as\'i por el estilo. Por
ejemplo, la compa\~n\'ia francesa LVMH-Moet Hennessy Louis Vuitton tiene diferentes estrategias a nivel de negocios para su divisi\'on
de alta costura Christian Dior, su divisi\'on de art\'iculos de cuero Louis Vuiltton, su divisi\'on de perfumes Guerlain, su divisi\'on de joyer\'ia
Fred Joailler, su divisi\'on del Co\~nac Hennessy y sus dem\'as divisiones de productos de lujo. Cada una de ellas ha desarrollado su
propio enfoque \'unico para distinguirse de sus competidores, mediante la identificaci\'on de sus clientes objetivos, productos
apropiados y promociones eficaces.

Cuando una organizaci\'on participa en diversos negocios, la tarea de la planificaci\'on se puede facilitar creando unidades estrat\'egicas
de negocios. Una unidad estrat\'egica de negocios (UEN) es un solo negocio o una colecci\'on de negocios de car\'acter
independiente, para los cuales se formula una estrategia propia. Representa un solo negocio o un grupo de negocios relacionados
entre s\'i. Cada UEN cuenta con su propia misi\'on \'unica, sus competidores y su estrategia. Estos rasgos permiten distinguir a una
UEN de los dem\'as negocios de la organizaci\'on matriz. En una compa\~n\'ia como General Electric, que participa en muchas l\'ineas de
negocios diferentes, es posible que los gerentes lleguen a crear una docena de UEN o m\'as.
 El concepto UEN separa las unidades de negocio tomando como base los siguientes principios:
\begin{itemize}
	\item La organizaci\'on es administrada por una "cartera" de negocios; cada unidad de negocios trabaja como un segmento de
      productos-mercado claramente definido, con una estrategia tambi\'en claramente definida.
	\item Cada unidad de negocios incluida en la cartera, desarrolla una estrategia a la medida de sus capacidades y necesidades
      competitivas, pero consistente con las capacidades y necesidades de la organizaci\'on en general.
	\item La cartera total es administrada en provecho de los intereses de la organizaci\'on en conjunto. Su prop\'osito es lograr un
      crecimiento equilibrado de las ventas y ganancias, y combinar los activos en un nivel de riesgo aceptable y controlado.
\end{itemize}
 El minorista franc\'es de productos de lujo LVMH-Hennessy Louis Vouitton constituye un excelente ejemplo del uso de estrategias
exitosas a nivel de negocios. El presidente de la firma, Bernard Arnault, supervisa las unidades estrat\'egicas de negocios que
manufacturan art\'iculos de cuero, perfumes, joyer\'ia, champa\~na y co\~nac.

	\item \emph{Estrategia a nivel funcional}

	Consiste, en una primera aproximaci\'on, en determinar la forma de respaldar a la estrategia a nivel de negocios.

	Cuando se trata de organizaciones que tienen departamentos funcionales de tipo tradicional, tales como producci\'on, marketing,
recursos humanos, investigaci\'on y desarrollo, y finanzas, esas estrategias tienen que servir de apoyo para la estrategia a nivel de
negocios. Por ejemplo, cuando la compa\~n\'ia del ramo de imprentas R. R. Donnelley \& Sons Company, con sede en Chicago, tom\'o la
decisi\'on estrat\'egica de invertir en suma apreciable en nuevos m\'etodos digitales de alta tecnolog\'ia para impresi\'on, su departamento
de marketing tuvo que desarrollar nuevos planes de ventas y elementos promocionales, el departamento de producci\'on procedi\'o a
incorporar equipo digital a las plantas impresoras, y el departamento de recursos humanos tuvo que actualizar sus programas para
la selecci\'on y capacitaci\'on de empleados.

Los t\'opicos a abordar en este texto, estar\'an orientados principalmente hacia las estrategias a nivel corporativo y a nivel de negocios.
Este \'enfasis no debe restar importancia a las estrategias a nivel funcional; m\'as bien, es un reflejo de la importancia que los
investigadores y los profesionales han dado al desarrollo de marcos estrat\'egicos.
	\end{itemize}


	\item ?`Qu\'e es la matriz BCG?

	La Matriz BCG, es una herramienta estrat\'egica para guiar las decisiones sobre asignaci\'on de recursos tomando como base
	la participaci\'on de mercado y el crecimiento de las UEN (Unidad Estrat\'egica de Negocios)
	La matriz BCG introdujo la idea de que cada una de las UEN de una organizaci\'on pod\'ia evaluarse y graficarse por medio de
	una matriz de 2 x 2 para saber cuales de ellas ten\'ian alto potencial y cuales eran solo un desperdicio de recursos organizacionales.
        La matriz BCG define cuatro grupos de negocios:
	\begin{itemize}
		\item \emph{Vacas de efectivo:} Son Negocios que muestran bajo crecimiento, pero les corresponde una alta participaci\'on de mercado.
			Los negocios de esta categor\'ia generan grandes cantidades de efectivo, pero sus perspectivas de crecimiento futuro son limitadas.
		\item \emph{Estrellas:} Son negocios que muestran alto crecimiento y una alta participaci\'on de mercado. Estos negocios est\'an en un mercado
			de r\'apido crecimiento y tienen una participaci\'on dominante en ese mercado, pero podr\'ian producir un flujo de efectivo positivo o no
			producirlo, seg\'un sus necesidades de invertir en nuevas plantas y equipo o en el desarrollo de productos.
		\item \emph{Interrogaciones:} Son negocios que muestran alto crecimiento, pero a la vez una baja participaci\'on de mercado. Estos negocios
			son especulativos e implican altos riesgos. Est\'an en una industria atractiva, pero les corresponde un peque\~no porcentaje de
			participaci\'on de mercado.
		\item \emph{Perros:} Son negocios que muestran un bajo crecimiento y tambi\'en una baja participaci\'on de mercado. Los negocios de esta
			categor\'ia no producen mucho efectivo, ni tampoco lo requieren en gran cantidad. Estos negocios no prometen mejor\'ia alguna en
			su rendimiento.
	\end{itemize}

	\item ?`Qu\'e es estrategia competitiva?

	Con el prop\'osito de crear una ventaja competitiva, las organizaciones buscan la forma de distinguirse de las dem\'as. Y lo hacen a
	trav\'es de las estrategias competitivas. La forma que una organizaci\'on elige para lograr ese prop\'osito es la esencia de la estrategia a
	nivel de negocios.
	Seg\'un Porter, un buen punto de partida para establecer una estrategia competitiva, es abordar el an\'alisis de la industria en donde se
	desenvuelve una organizaci\'on.
	La clave consiste en explotar una ventaja competitiva.
	En cualquier industria, cinco fuerzas competitivas son las que determinan las reglas de competencia.
	Los gerentes eval\'uan el atractivo de una industria en funci\'on de estos cinco factores:
	\begin{itemize}
		\item \emph{Amenaza de nuevos competidores y barreras de entrada:} Factores tales como las econom\'ias de escala, la lealtad de la marca y
			los requisitos de capital, determinan el grado de facilidad o dificultad que tendr\'an los nuevos competidores para entrar a una
			industria.
		\item \emph{Amenaza de sustitutos:} Factores tales como los cambios de costos y la lealtad de los compradores determinan el grado en que
			los clientes estar\'an dispuestos a comprar un producto sustituto.
		\item \emph{Poder de negociaci\'on de los compradores:} Factores tales como el n\'umero de compradores en el mercado, la informaci\'on con
			que cuentan dichos compradores y la disponibilidad de sustitutos determinan el grado de influencia que tienen los compradores
			en la industria
		\item \emph{Poder de negociaci\'on de los proveedores:} Factores tales como el grado de concentraci\'on de los proveedores y la disponibilidad
			de insumos sustitutos, determinan la cantidad de poder que tienen los proveedores sobre las firmas de la industria.
		\item \emph{Presencia de rivalidades:} Factores tales como el crecimiento de la industria, el aumento o disminuci\'on de la demanda, y las
			diferencias entre productos, determinan cu\'an intensa ser\'a la rivalidad competitiva entre las diversas firmas de la industria.
			Selecci\'on de una estrategia competitiva.         En opini\'on de Porter, ninguna firma puede tener un rendimiento exitoso a un nivel
			superior al promedio si se propone destacarse en todo frente a toda la gente. Él propone que los gerentes seleccionen la estrategia
			que proporcione a la organizaci\'on una ventaja competitiva. Porter agrega que una ventaja competitiva se obtiene ya sea porque una
			firma tiene costos m\'as bajos que la competencia o porque es apreciablemente distinta de sus competidores. Sobre esa base, los
			gerentes pueden escoger una de estas tres estrategias: liderazgo de costos, diferenciaci\'on o enfoque. La selecci\'on que hagan los
			gerentes depender\'a de las fortalezas y las competencias fundamentales de la organizaci\'on, y de las debilidades de sus competidores.
	\end{itemize}



\end{enumerate}
