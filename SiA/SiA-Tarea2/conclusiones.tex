En primera instancia, podemos concluir que las organizaciones no son entidades est\'aticas
que nos permiten relacionarnos para lograr nuestros deseos individuales y/o grupales. M\'as bien,
una organizaci\'on es un conjunto de consensos sociales definidos bajo el contexto de nuestra propia
historia, es decir, son declaraciones lingu\'isticas que recopilan pensamientos acordes a la \'epoca
en que se viva. Lo que tuvo sentido hace 1000 a\~nos ya no lo tiene, por tanto la definici\'on de una
organizaci\'on est\'a sujeta a constantes cambios. La gracia est\'a en poder llegar a la ra\'iz de lo que se
entiende por organizaci\'on para as\'i poder redise\~narlo de acuerdo al contexto, o lo que menciona el
texto como ``reconstrucci\'on ontol\'ogica''.\\

Se entiende que una organizaci\'on es un fen\'omeno pol\'itico, enti\'endase como un fen\'omeno
dado por la conversaci\'on social, referida a lograr los deseos propios y globales. Luego una organizaci\'on
es una conversaci\'on, pero no viceversa.\\

El hilo conductor dentro de una organizaci\'on es el lenguaje, sin \'el, nada podr\'ia acontecer, es
por esto que se debe comprender las estructuras de lenguaje dado un cierto contexto. Enti\'endase
al lenguaje como una dicotom\'ia, es decir, un constante hablar y escuchar, donde se reconocen partes
b\'asicas como la petici\'on que es respondida por la promesa, ya sea de una acci\'on o de una posibilidad.\\

 El escuchar no se refiere a ``o\'ir'' lo que pasa a mi alrededor, sino m\'as bien, a la percepci\'on y
juicio que yo pueda emitir dado el contexto en el que me desenvuelvo. Haciendo esta parte del
lenguaje en la persona como individuo que ha sido criado con ciertos patrones de entendimiento.\\

Todo este proceso es inmensamente rico en dos extremos opuestos: el trasfondo en que se
desarrolla (historia) y las posibilidades que se dan (pensamiento individual, donde influyen los juicios
 sociales).\\

%       Apreciaciones Personales

El texto presenta bastantes apreciaciones de \'indole filos\'ofica, mencion\'andose a grandes
nombres como Heidegger, del cual poca informaci\'on poseemos. Este matiz permite dar un espectro
 m\'as grande de explicaci\'on del pensamiento de Flores, dado que no est\'a creando ideas nuevas,
sino que se apoya en criterios ya hechos y probados para generar una nueva apreciaci\'on.\\

Se puede ver que las personas estamos mal enfocadas en muchos aspectos de nuestro pensar,
 dado que es muy dogm\'atico desde que somos ni\~nos. Nos forman de ciertas maneras con ciertas
estructuras mentales que luego hacemos propias y no podemos romper, de ah\'i que aceptemos
un ambiente de trabajo desagradable o un bajo sueldo, creemos que as\'i es la organizaci\'on pues
nunca nadie nos ense\~n\'o como se defin\'ia y por ende no podemos redise\~narla. Esto tiene directa
relaci\'on con el texto ``La quinta disciplina'' donde se mencionan los modelos mentales.\\

Finalmente podemos apreciar que Flores ha extendido a Senge agregando a su comprensi\'on
de la organizaci\'on la arista del lenguaje.\\

Nos pareci\'o un texto realmente interesante en todo aspecto. Rescata aspectos de la vida
que son cotidianos y que la mayor\'ia del tiempo no reparamos en su importancia. Sin duda una recopilaci\'on
de escritos que a m\'as de un miembro de directorio le debe haber abierto los ojos, ya
que si bien los temas metalingu\'isticos suelen ser dif\'iciles de entender, hay una verdad de fondo que
es indiscutible: nada ocurre en las organizaciones sin el lenguaje, hablar y escuchar es una tarea
fundamental.
\vspace{15cm}
