Los seres humanos viven en organizaciones, en el d\'ia a d\'ia nos envolvemos en un trabajo que es posible gracias a marcos organizacionales. El ser social m\'as significativo es un comportamiento organizacional. Lamentablemente no se comprende que es una organizacion, ni su dise~no y en que pueden evolucionar. Las persona que se envuelven en organizaciones no act\'uan de manera creativa a los problemas que se le presentan y se resignan ante el fracaso de algun proyecto organizacional. La respuesta es de \'ambito intelectual como pr\'actico. \\ 

Una organizaci\'on se define como un lugar donde se generan conversaciones, las cuales son fen\'omenos sociales en los que se realizan trabajos como toma de desiciones, formacion de juicios y manejo de posibilidades. Las organizaciones son el producto de las conversaciones y no viceversa. \\ 

Para la definici\'on de una organizaci\'on debemos declarar ciertos oficios que cubran el amplio campo de las conversaciones y posibles movimientos de ellas, ademas de definir quien esta facultado y tiene el deber de realizar los oficios. \\

Estas declaraciones en s\'i no son los objetivos o metas, sin embargo, las organizaciones estan basadas en conversaciones, tanto recurrentes como de un trasfondo espec\'ifico y compartido. \\

Las organizaciones deben plantearse proyectos espec\'ificos, los cuales poseen un camino abierto gracias a la declaraci\'on de su dominio de posibilidades. Estos proyectos requieren un lenguaje de acci\'on y condiciones de satisfacci\'on para su realizaci\'on. \\

	CondicionesSatisfechas($accion_1$, $accion_2$, ..., $accion_n$)\\

La estructura interna de una organizaci\'on no basta con ser representada por un organigrama, sino que se deben especificar las relaciones entre las personas, las cuales consisten en compromisos lingu\'isticos entre ellas. Estos compromisos implican peticiones, promesas, declaraciones o afirmaciones entre personas, descubriendo de esta forma la estructura real de una organizacion. \\

En una organizaci\'on el poder se encuentra localizado, lo que implica problemas si es que en un sector se ubica la tarea de crear y administrar las conversaciones y en otro sector el declarar el \'ambito de posibilidades para la organizaci\'on, debido a que puede ocurrir que las acciones demandadas no se encuentren en entre las posibilidades declaradas. Lo importante es tener en cuenta las declaraciones a nivel operacional y no las que se anuncian para una acci\'on especifica, adem\'as de tener presente de que posibilidades permite la estructura de la organizaci\'on para ejecutar el poder, que posibilidades de poder existen considerando a la organizacion con su entorno, historia, cultura y pol\'itica, posibilidades de poder tiene la organizaci\'on considerando su pasado o herencia y su futuro o destino. \\

La emoci\'on es m\'as que un fen\'omeno natural. Las emociones definen nuestro estado de \'animo, debido a \'esto, se produce un fen\'omeno lingu\'istico que conlleva a producir variaciones en nuestro espacio de posibilidades frente al futuro. El estado de \'animo y juicios expanden o restringen la gama de conversaciones que podemos establecer con otras personas. Estas posibilidades de apertura y cierres se producen al escuchar. \\

Es esencial, si vamos a dirigir nuestras vidas y nuestras organizaciones como posibilidades, que escuchemos los estados de \'animo, las emociones y los juicios como aperturas y cierres de posibilidades sociales. \\

\vspace{14cm}
