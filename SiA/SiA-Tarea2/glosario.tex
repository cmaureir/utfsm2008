\begin{itemize}
	\item \emph{Conversaci\'on:} Proceso lingu\'istico en donde alguien hace una petici\'on y la contraparte
		realiza una promesa, adecu\'andose a la petici\'on mediante las bien llamadas “condiciones de satisfacci\'on”.
		Nada de esto ocurre sin el lenguaje, sin hablar y escuchar.
	\item \emph{Condiciones de Satisfacci\'on:} Son aquellas condiciones a cuales se somete el cumplimiento de una promesa
		o petici\'on, acordada en una conversaci\'on.
	\item \emph{Afirmaci\'on – Declaraci\'on:} Afirmaci\'on, es darle el valor de verdad a algo dicho, en cambio, declaraci\'on
		no apunta a decir como son las cosas, sino a hacer que sean de esa forma.
	\item \emph{Teor\'ia de la Acci\'on:} Teor\'ia que afirma que toda acci\'on humana esta precedida por actos lingu\'isticos
		que la determinan.
	\item \emph{Workflow:} Flujo de Trabajo. Secuencia de procesos industriales, administrativos u otros a trav\'es de los cuales
		una tarea pasa de ser iniciada a ser completada
	\item \emph{Retroalimentaci\'on:} Informaci\'on acerca de las reacciones de un individuo ante un est\'imulo. Usado como
		base para mejoras.
	\item \emph{Metalingu\'istico:} Lenguaje que se usa para explicar o hablar del lenguaje mismo.
\end{itemize}
