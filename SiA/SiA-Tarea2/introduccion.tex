Cuando ni\~nos nos ense\~nan que la comunicaci\'on es el acto de entregar y recibir datos entre
personas, conversar acerca del clima, preguntarnos como nos ha ido en el d\'ia, saludarnos para los
cumplea\~nos, escribirnos correos, llamarnos por tel\'efono, etc. y desde entonces hemos dado por
sentado sin un menor esfuerzo por indagar al respecto que en eso se agota la comunicaci\'on. Nos
desenvolvemos en un mundo en el que al parecer la relevancia de los actos lingu\'isticos quedan en
un segundo plano donde se le atribuye tan s\'olo el m\'erito de permitir a los humanos expresarse e
intercambiar ideas y sentimientos. Parece una percepci\'on bastante razonable, sin embargo veremos 
que la comunicaci\'on a trav\'es del lenguaje implica una construcci\'on de la realidad en esa acci\'on
 de entregar y recibir datos.\\

En el presente informe abordaremos temas presentes en el texto de Fernando Flores como
la Teor\'ia de la Acci\'on y los respectivos conceptos que la conforman como el lenguaje, las peticiones y
promesas, y las conversaciones que dentro de \'este se dan. Destacaremos la importancia que tienen
acciones simples como escuchar para colaborar y planificar siendo creativos para el buen funcionamiento
 de empresas y organizaciones en general. Caeremos en la cuenta de que la comunicaci\'on
es en la pr\'actica un sistema de informaci\'on.\\

Estudiaremos el concepto de "Workflow" (Flujo de Trabajo) demostrando que la coordinaci\'on
de masas se realiza a trav\'es de la conversaci\'on. Veremos como los estados de \'animo y emociones
juegan un papel absolutamente importante en el dise\~no de las organizaciones. Profundizaremos
acerca del acto de escuchar, que hoy en d\'ia creemos entender pero no siempre lo entendemos del
todo.\\

Retomaremos lo estudiado previamente en "La Quinta Disciplina" por Peter Senge, donde
aprendimos que el aprendizaje en una organizaci\'on es totalmente imprescindible y debe ser constante
 y permanente en el tiempo, de modo que nuestra empresa cumpla sus objetivos globales y
los de sus miembros.
\vspace{5cm}
