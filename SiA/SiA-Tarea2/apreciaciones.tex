\begin{itemize}
	\item ?`Qu\'e diferencias existen entre organizaciones y conversaciones?

No existe duda que las organizaciones son fen\'omenos pol\'iticos, pero como bien sabemos que las organizaciones
son el producto de conversaciones, acciones y demases. Por ende Las organizacione tambi\'en producto de nuestras
conversaciones sobre como tendremos conversaciones acerca del contexto social (instituciones, oficinas, reglamentos,
etc) dentro del cual sostendremos conversaciones.
?`Pero es f\'acil poder distinguir que conversaciones son pol\'iticas?

Con algunos t\'erminos pol\'iticos como Democracia, Socialismo, Monarqu\'ia, etc, van a se\~nalar la posici\'on uno puede tomar con
respecto a alg\'un hecho, es decor, el camino que va a seguir la conversaci\'on organizacional. Si pudiesemos comprender todos los
t\'erminos, podriamos saber a trav\'es de ciertas declaraciones, como se transmite la fuerza de la acci\'on de la persona que las
emite, aparte de quienes pueden participar en ciertas conversaciones relacionadas con los t\'erminos anteriormente se\~nalados.
Si adoptamos \'estas posiciones, producimos los arreglos sociales esenciales para las conversaciones que podemos tener a futuro.
Tenemos que tener presente la idea de que somos todos parte de las conversaciones pol\'iticas, herando declaraciones que ya se han
dicho, por lo cual sabemos como se han desarrollado dichas conversaciones en nuestra sociedad.
Finalmente me gustaria compartir \'este parrafo, con el cual flores, nos deja se\~nalado como las conversaciones se convierten en un
ciclo que nosotros misnos formamos y aportamos:\\

\begin{center}
\emph{Incluso, cuando nuestra tarea es producir una organizaci\'on nueva, vale decir, una nueva compa\~n\'ia
o incluso una organizaci\'on pol\'itica del Estado enteramente nueva, nosotros s\'olo nos volvemos a unir a una conversaci\'on en la cual ya hemos
participado en calidad de escuchadores. De este modo el dise\~no de las organizaciones nunca parte completamente de nuevo.}
\end{center}

	\item El fenomeno ling\"uistico\\\\
 Es curioso como Fernando Flores se dio cuenta que del conversar (en el contexto organizacional) nac\'ia absolutamente todo.\\
 Encontramos que su forma de ver las cosas, el identific\'o que la organizaci\'on no es un conjunto de individuos sino una red
 de conversaciones, es bastante \'util y cierto. Especialmente, porque constituye que la empresa sustancialmente no son seres 
humanos, sino redes de conversaciones entre ellos. Es decir, de esta forma podemos identificarlas como redes de compromisos 
ling\"u\'isticos, redes de actos del habla que nos permiten estudiar mas a fondo las distintas interacciones que los representan
 en realidad. Esta postura, encontramos que es muy interesante de analizar, ya que sustenta la conversaci\'on como ``la unidad 
m\'inima de interacci\'on social orientada hacia la ejecuci\'on con \'exito de acciones''. De este modo,
 la conversaci\'on se convierte en un fen\'omeno clave en las organizaciones, partiendo de la base que el lenguaje es invenci\'on y 
constituci\'on de realidad. El \'enfasis del concepto est\'a centrado en la comunicaci\'on para la acci\'on. Y gracias a esto, uno
 puede darse cuenta de como para entender una organizac\'ion (o conocerla realmente), es m\'as relevante estudiar las diferentes redes
 de conversaciones que ven resultados concretos, sin importar el tipo de empresa o organizacion, ya sea en \'ambitos de los negocios,
 de la educaci\'on, del tiempo libre, de las finanzas o de la pol\'itica.\\

As\'i, compartimos la opinion que el lenguaje, m\'as que ser una herramienta descriptiva, se vuelve una pr\'actica articuladora de los
 diferentes procesos que se llevan a cabo. El que el lenguaje forma un constructor de realidad y forma en que la historia se manifiesta.
 A pesar de no ser algo muy facil de comprender, tambien entendemos que nada ocurre dentro de la organizaci\'on sin el lenguaje. Sin 
lenguaje no podr\'iamos construir organizaciones. A trav\'es de \'el, los individuos se transforman en miembros del entorno ps\'iquico 
de la organizaci\'on. La piensan desde all\'i y logran definirla en su totalidad.\\

	\item Sobre su pensamiento

Se entiende de que Fernando Flores  no est\'a interesado en la organizaci\'on en s\'i, 
como un ente, por el contrario hace referencia a los problemas que se producen en la realizaci\'on y dese~no del trabajo.

Flores se~nala un pensamiento, que da el pie inicial sobre una nueva visi\'on al analizar las organizaciones, como una especie de redes de compromisos entre las personas que las componen, las cuales se atribuyen a acciones de caracter lingu\'istico.

Tambi\'en menciona una definici\'on para los tipos de procesos que definen el funcionamiento de las organizaciones: \\

Un proceso de producci\'on, en donde se refiere a transformar y unir las materias primas en componentes y productos. \\ \\
Luego se refiere a la producci\'on de informaci\'on, el cual consiste en manipular, comunicar y despleguar eventos y procesos referentes al negocio. \\ \\
Por \'ultimo hace referencia a un proceso de satisfacci\'on al cliente, el cual se refiere a realizar y completar condiciones de satisfacci\'on entre los clientes y el que provee.

\end{itemize}
\vspace{5cm}
