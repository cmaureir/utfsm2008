\begin{enumerate}
	\item Escuchar

Flores se\~nala que si una persona establece en una conversaci\'on de que puede
 ocurrir un problema si no se toma alguna desici\'on anticipada, este problema
 es algo subjetivo en la persona, aun cuando puede que otra persona no interprete
 en su escuchar un aparente problema.

Hablar y escuchar son fen\'omenos ricos segun Flores. Esto debido a que una
 persona si bien puede esta comunic\'andonos algun tipo de informaci\'on,
 nosotros obviamente lo escuchamos, pero a su vez el tambien se escucha, 
lo que implica que tiene en cuenta todas las cosas que el traspaso de esta
 informaci\'on significa, ya sea compromisos de hacer algo nuevo o compromisos
 personales para la realizaci\'on de ello. El escuchar implica el transfondo 
de cualquier cosa que le ocurra.

En momento en que se hace presente el acto de hablar es apenas la punta del 
iceberg en relaci\'n con este trasfondo. Esto implica compromisos espec\'ificos
 y expl\'icitos que el que comunica puede haber adquirido respecto a la nueva
 tarea que se gener\'o o generara, resultado de la informaci\'on que comunic\'o.


	\item El fen\'omeno ling\"u\'istico en las organizaciones

Seg\'un Fernando Flores, en las organizaciones, todo ocurre en base al
lenguaje, o dicho de mejor modo, a las conversaciones.\\
En toda conversaci\'on, alguien hace una petici\'on y la contraparte
(recordemos que en una conversaci\'on participa m\'as de una persona) realiza
una promesa, adecu\'andose a la petici\'on mediante las bien llamadas
``condiciones de satisfacci\'on''. Nada de esto ocurre sin el lenguaje, sin hablar y
escuchar. La organizaci\'on misma no existir\'ia sin el lenguaje, sin que alguien la
haya propuesto, anunciando p\'ublicamente su existencia.\\

Es de vital importancia aclarar qu\'e es lo que realmente se entiende por
lenguaje, por lenguaje entendemos conversaci\'on, pero para ser m\'as precisos,
``conversaciones para la acci\'on'' y ``conversaciones de posibilidades''. Pero
entonces, ?`qu\'e entendemos por conversaciones para la acci\'on?. Las
conversaciones para la acci\'on son aquellas mediante las cuales logramos que
las cosas se realicen, a diferencia de las conversaciones de posibilidades, las
cuales surgen a ra\'iz de las conversaciones para la acci\'on.\\

Al hacerse la petici\'on con su correspondiente promesa, aparece lo que
denominamos ``condiciones de satisfacci\'on'', que no es otra cosa que las
condiciones bajo las que se cumplir\'a una petici\'on o una promesa.
En lo que va del texto ya hemos profundizado dos grandes conceptos
que son posibles distinguir en las conversaciones para la acci\'on. Ahora se
ver\'an otros dos conceptos: las afirmaciones y las declaraciones.\\

Con una afirmaci\'on se dice que algo es as\'i o verdadero, una declaraci\'on,
por el contrario, no es decir que algo es as\'i: es hacer que sea as\'i. Las
conversaciones para la acci\'on comprometen a actuar; en cambio las conversaciones
 para posibilidades producen oportunidades para comprometerse en una acci\'on.\\

	\item El rol de la planificaci\'on

Dado que las organizaciones deben estar preparadas para cualquier eventualidad futura, \'estas
deben realizar extensos y rigurosos procesos de planificaci\'on para poder responder de la mejor
manera a las inclemencias del futuro. Hoy en d\'ia planificar significa prever, hacer predicciones y
estimaciones acerca de los d\'ias venideros a tal punto que nos preparamos para un futuro que creemos
va a suceder inminentemente. En este proceso lamentablemente estamos dejando de innovar.\\

No nos referimos a innovar como un proceso de descubrir una f\'ormula del \'exito completamente
revolucionaria. La creatividad en este caso pasa por un asunto de saber escuchar posibilidades
para nuestra organizaci\'on de avanzar hacia el desarrollo. Muchas veces empresas con gran
capacidad t\'ecnica y buenos profesionales no triunfan por el simple hecho de que no saben extraer
de la realidad el camino a seguir. Un contra ejemplo al respecto es el siguiente: empresas como
Apple Computer sin tener mayores ventajas en avances tecnol\'ogicos (los mismos que su competencia
en aquel tiempo) termin\'o siendo un gigante de la computaci\'on personal a nivel mundial \'unicamente
gracias a la visi\'on emprendedora de sus miembros. Vale decir, supieron escuchar.\\

Como mencionamos, la comunicaci\'on se puede representar como un sistema de informaci\'on
en el cual las peticiones y ofertas son los mensajes que se env\'ian los distintos componentes. En
este caso los componentes son personas y las interacciones que se dan entre ellos corresponden a
conversaciones.



\end{enumerate}
\vspace{8cm}
